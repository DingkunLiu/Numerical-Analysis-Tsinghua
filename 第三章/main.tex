%!TEX program=xelatex
\documentclass[a4paper]{article}

\usepackage{amsmath}
\usepackage{amsthm}
\newtheorem{definition}{定义}[section]
\newtheorem{theorem}{定理}[section]
\newtheorem{lemma}{引理}[section]
\newtheorem{example}{例}[section]

\makeatletter % `@' now normal "letter"
\@addtoreset{equation}{section}
\makeatother  % `@' is restored as "non-letter"
\renewcommand\theequation{\oldstylenums{\thesection}%
                   .\oldstylenums{\arabic{equation}}}


\usepackage{ctex}
\usepackage{extarrows}
\usepackage{amssymb}
\usepackage{tabularx}
\usepackage{multicol}
\usepackage{multirow}
\usepackage{booktabs}
\usepackage{hyperref}

\title{第三章\ 非线性方程和方程组的数值解法}
\author{}
\date{}

\begin{document}
\maketitle

$f(x)=0$的解也叫根或$f(x)$的零点。\\
若$f(x)=0(x\in [a,b])$有解,则称$[a,b]$为有根区间,若$f(x) \in C[a,b]$,且$f(a)f(b)<0$在$[a,b]$至少存在一实根。

\section{二分法}
过程略。\\
$\because b_2-a_2 = \frac{b-a}{2}$,由此易证$b_n-a_n=\frac{b-a}{2^{n-1}}$;且$x_n, x^* \in [a_n, b_n]$\\
$\therefore |x_n-x^*|<\frac{b_n-a_n}{2}=\frac{b-a}{2^n}<\varepsilon, \therefore n > \frac{ln\frac{b-a}{\varepsilon}}{ln2} $

\section{单个方程的不动点迭代法}
\subsection{不动点迭代法}
\begin{equation}
    f(x)=0, x\in [a,b]
    \label{eq:2.1}
\end{equation}
将(\ref{eq:2.1})转化为如下等价的方程:
\begin{equation}
    x=\phi(x), \phi(x)\in [a,b]
\end{equation}
由此可得如下迭代公式:
\begin{equation}
    x_{k+1} = \phi(x_k), k=0,1,2,\dots
    \label{eq:2.3}
\end{equation}
若$x_k$收敛,即$x_k\rightarrow x^*, k\rightarrow \infty$则$x^*=\phi(x^*)$.$\phi(x)$称为迭代函数,$x^*$称为$\phi(x)$的不动点。

\begin{theorem}
    \textbf{!!}(压缩映射原理)设$\phi(x)\in C[a,b]$,若满足下面条件:
    \begin{enumerate}
        \item 映内性:\begin{equation}
            \phi(x) \in [a,b], \forall x\in [a,b]
            \label{eq:2.4}
        \end{equation}
        \item 压缩性(Lipschitz条件):$\exists 0\le L <1$使得:
        \begin{equation}
            |\phi(x)-\phi(y)|\le L|x-y|, \forall x,y\in [a,b]
            \label{eq:2.5}
        \end{equation}
    \end{enumerate}
    则$\forall x_0\in[a,b]$,由\ref{eq:2.3}产生的迭代序列$\{x_k\}$都收敛到$\phi$的\textbf{唯一}不动点,且有误差估计式:
    \begin{equation}
        |x_k-x^*|\le \frac{L^k}{1-L}|x_1-x_0|
        \label{eq:2.6}
    \end{equation}
    \label{theorem:2.1}
\end{theorem}

\begin{proof}
    分四步: (1)不动点存在(2)唯一性(3)收敛性(4)误差估计

    \begin{enumerate}
        \item 令$f(x)=x-\phi(x)$,则由\ref{eq:2.4}可得$f(a)=a-\phi(a)\le 0, f(b)=b-\phi(b)\ge 0,\therefore \exists x^* \in [a,b], s.t. f(x)=0, $
        i.e.$x^*-\phi(x^*)=0\Leftrightarrow x^*=\phi(x^*)$
        \item 唯一性[反证法]:设$x_1^*=\phi(x_1^*),  x_2^*=\phi(x_2^*), x_1^*\neq x_2^*, x_1^*, x_2^*\in [a,b]$
            则$| x_1^*- x_2^*|=|\phi( x_1^*)-\phi( x_2^*) |\le L| x_1^*- x_2^*|<| x_1^*- x_2^*|$矛盾,$\therefore$唯一。
        \item 收敛性:$\because \phi(x)\in [a,b], \forall x\in [a,b]$\\
        $\therefore x_k\in [a,b], k=0,1,2,\dots$\\
        $\therefore |x_k-x^*|=|\phi(x_{k-1})-\phi(x^*) |\le L|x_{k-1}-x^*|\le \dots \le L^k|x_0-x^*|\rightarrow 0, k\rightarrow \infty$\\
        $\therefore \lim_{k\rightarrow \infty}x_k=x^*$
        \item 误差估计式:
        $\forall m$, 
        \begin{equation}
            \begin{split}
                |x_{k+m}-x_k |&=|x_{k+m}-x_{k+m-1}+x_{k+m-1}-x_{k+m-2}+x_{k+m-2}+\dots+x_{k+1}-x_k | \\
        &\le|x_{k+m}-x_{k+m-1}|+|x_{k+m-1}-x_{k+m-2}|+\dots+|x_{k+1}-x_k |
            \end{split}
            \label{eq:*1}
        \end{equation}
        而$|x_{k+j+1}-x_{k+j} |=|\phi(x_{k+j})-\phi(x_{k+j-1})|\le L|x_{k+j}-x_{k+j-1} |\le \dots \le L^{k+j}|x_1-x_0|$\\
        $\therefore m\rightarrow \infty, |x_k-x^*|\le \frac{L^k}{1-L}|x_1-x_0|$
    \end{enumerate}
\end{proof}

注释:
\begin{enumerate}
    \item 由\ref{eq:*1}式易证一个误差估计式$|x_k-x^*|\le \frac{L}{1-L}|x_k-x_{k-1}|$,当L不接近于1时,可用$|x_k-x_{k-1}|<\varepsilon$作为终止条件。
    \item 可用\ref{eq:2.6}估计迭代次数:
    $$k\ge \frac{ln\varepsilon-ln(|x_1-x_0|/(1-L))}{lnL}$$
    \item 当L越小,收敛越快
    \item 满足定理条件的迭代法是全局收敛的。
\end{enumerate}

推论:若将定理\ref{theorem:2.1}中的条件(2)改为$\phi(x)$的导数在$[a,b]$上存在,且$\phi^{'}(x)\le L<1, \forall x\in[a,b]$则定理\ref{theorem:2.1}的结论仍成立。

\begin{example}
    解方程$f(x)=x^3+2x^2-4=0, x^*\approx 1.130395$
    易证$f(x)=0$在$[1,2]$上有根,$f(1)<0,f(2)>0$;将$f(x)=0$改写为如下不同形式:
    \begin{enumerate}
        \item $$x=x-x^3-2x^2+4\overset{\Delta}{=} \phi_1(x)$$
        \item $$x=[2(\frac{2}{x}-x)]^{\frac{1}{2}} \overset{\Delta}{=} \phi_2(x)$$
        \item $$x=(2-\frac{x^3}{2})^\frac{1}{2}\overset{\Delta}{=} \phi_3(x) $$
        \item $$x=2\sqrt{\frac{1}{x+2}}\overset{\Delta}{=} \phi_4(x)$$
        \item $$x=x-\frac{x^3+2x^2-4}{3x^2+4}\overset{\Delta}{=} \phi_5(x)$$
    \end{enumerate}
    
    由此得到的迭代关系式,1和2不收敛,3需要120次收敛,4需要9次迭代,5需要4次迭代。
\end{example}

\subsection{局部收敛性和收敛阶}
\begin{definition}
    设$\phi(x)$在$[a,b]$区间有不动点$x^*$,若$\exists x^*$的一个邻域$R(x^*)=[x^*-\varepsilon_0, x^*+\varepsilon_0]\subset [a,b] $,
    $\varepsilon_0>0$使对$\forall x_0\in R(x^*)$,由$x_{k+1}=\phi(x_k) $产生的迭代序列$\{x_k\}\subset R(x^*)$且收敛于$x^*$,则称此迭代方法局部收敛。
\end{definition}

\begin{theorem}
    设$x^*$为$\phi(x)$的不动点,$\exists R(x^*)$,$\phi(x)\in C^1[R(x^*)]$,且$|\phi^{'}(x^*)|<1$,则$x_{k+1}=\phi(x_k) $局部收敛。
\end{theorem}
\begin{proof}
    $\because \phi^{'}(x)$连续,且$|\phi^{'}(x)|<1, \therefore$由微积分定理可知:$\exists x^*$的一个邻域$R(x^*)=[x^*-\varepsilon_0, x^*+\varepsilon_0]$,使得$|\phi^{'}(x)|\le L<1, \forall x\in R(x^*)$\\
    任取$x\in R(x^*)$,则$|\phi(x)-x^*|=|\phi(x)-\phi(x^*)|=|\phi(x)-x^*|=|\phi^{'}(\xi)|\cdot|x-x^*|\le L|x-x^*|<|x-x^*|\le \varepsilon$\\
    $\Rightarrow \phi(x)\in [R(x^*)]$,满足映内性。\\
    显然$\forall x, y\in R(x^*) $,$|\phi(x)-\phi(y)|=|\phi^{'}(\eta)||x-y|\le L|x-y|$,压缩性。\\
    $\therefore$由压缩映射原理可知,迭代法具有局部收敛性。
\end{proof}

\paragraph{思考题}
\begin{itemize}
    \item $x_{k+1}=sin(x_k), x^*=0, |\phi^{'}(x^*)|=1 $,任取$x_0$收敛
    \item $x_{k+1}=x_k+x_k^3, x^*=0, |\phi^{'}(x^*)|=1 $,发散
    \item $x_{k+1}=x_k+x_k^2, x^*=0, |\phi^{'}(x^*)|=1 $,$x_0>0$发散;$x_0<0$收敛
\end{itemize}
证明:$|e_{k+1} |=|x_{k+1}-x^* |=|\phi(x_k)-\phi(x^*)|=|\phi^{'}(\xi)||x_k-x^*|$

\begin{definition}
    设$\{x_k\}$收敛于$x^*$,记$e_k=x_k-x^*$,若$\exists$实数$p\ge 1$及非零常数C,使$\lim_{k\rightarrow\infty}\frac{e_{k+1}}{|e_k|^p}=C $,则称迭代序列$\{x_k\}$为p阶收敛的。

    $p=1$ 称为线性收敛;$p=2$称为平方收敛。

    $p=1$时,$0<c<1$;$p>1$,超线性收敛。
\end{definition}

如果$\phi(x)\in C^1[R(x^*)]$,则$\lim_{k\rightarrow\infty}\frac{|e_{k+1}|}{e_k}=\lim_{k\rightarrow\infty}\frac{|\phi(x_k)-\phi(x^*)|}{x_k-x^*}=|\phi^{'}(x^*)| \left\{
    \begin{array}{lr}
        \neq 0, <1 \ \text{线性}\\
        =0 \ \text{超线性}
    \end{array}
    \right. $

\begin{theorem}
    设$\phi(x)$有不动点$x^*, p>1,\phi(x)\in C^p[R(x^*)]$,且$\phi^{{'}{'}}(x^*)=\phi^{(3)}(x^*)=\dots = \phi^{(p-1)}(x^*)=0$,$\phi^{(p)}(x^*)\neq 0$,则称迭代法是p阶收敛的,且$\lim_{k\rightarrow\infty}\frac{|e_{k+1}|}{e_k}=\frac{\phi^{(p)}(x^*)}{p!}$
\end{theorem}
\begin{proof}
    $\because \phi^{'}(x^*)=0, \therefore$由局部收敛性定理可知$\forall x_0\in R(x^*) $收敛
    $$\phi(x_k)=\phi(x^*)+\phi^{'}(x^*)(x_k-x^*)+\dots+\frac{\phi^{(p-1)}(x^*)}{(p-1)!}(x_k-x^*)^{p-1}+ \frac{\phi^{(p)}(\xi)}{(p)!}(x_k-x^*)^{p}$$
    $\Rightarrow e_{k+1}=x_{k+1}-x^*= \frac{\phi^{(p)}(\xi)}{(p)!}e^p_k\Rightarrow \lim_{k\rightarrow\infty}\frac{e_{k+1}}{e_k^p}=\frac{\phi^{(p)}(x^*)}{p!}$
\end{proof}

\section{迭代加速方法}
\paragraph{Steffensen迭代法}
\begin{equation*}
    \left\{
        \begin{array}{lr}
            y_k=\phi(x_k)\\
            z_k=\phi(y_k)\\
            x_{k+1}=x_k-\frac{(y_k-x_k)^2}{z_k-2y_k+x_k}
        \end{array}
        \right.
\end{equation*}
其对应的迭代函数:
\begin{equation*}
    \begin{array}{lr}
        \Psi(x)=x-\frac{[\phi(x)-x]^2}{\phi[\phi(x)]-2\phi(x)+x}, |\phi^{'}(x^*)|<1 \\
        x_{k+1}=\Psi(x_k)
    \end{array}
\end{equation*}
易证若$\phi(x)\in C^1[R(x^*)]$,则$x^*=\Psi(x^*)\Leftrightarrow x^*=\phi(x^*) $

\begin{theorem}
    设$x^*=\phi(x^*)$,且$\phi^{(p+1)}(x)\exists, \forall x\in R(x^*) $,则当$p=1$时,若$\phi^{'}(x^*)\neq 1 $,那么该方法是二阶的,否则为一阶收敛的,当$x_{k+1}=\phi(x_k)$是p阶收敛的$(p>1)$,则该法是$2p-1$阶的。
\end{theorem}

\section{Newton迭代法}
如果取$\phi(x)=x-\frac{f(x)}{f^{'}(x)}$,则有如下Newton迭代方法:
\begin{equation}
    x_{k+1}=x_k-\frac{f(x_k)}{f^{'}(x_k)},k=0,1,\dots
    \label{eq:4.1}
\end{equation}
显然有$\phi^{'}(x)=\frac{f(x)f^{{'}{'}}(x)}{[f^{'}(x)]^2}\Rightarrow \phi^{'}(x^*)=0$\\
$\therefore $(\ref{eq:4.1})收敛。

\begin{theorem}
    设$f(x^*)=0,f^{'}(x^*)\neq 0 $[单根,不存在重根],且$f(x)\in C^m[R(x^*)], m\ge 2$则Newton迭代方法\ref{eq:4.1}至少2阶收敛,且$\lim_{k\rightarrow \infty}\frac{|e_{k+1}|}{e_k^2}=\frac{f^{{'}{'}}(x^*)}{2f^{'}(x^*)}$
\end{theorem}

\subsection{重根时的Newton迭代法及加速}
设$x^*$为$f(x)=0$的$m(m\ge L)$重根,i.e.$f(x)=(x-x^*)^mq(x),q(x^*)\neq 0 $,则
$\phi^{'}(x)=\frac{f(x)f^{{'}{'}}(x)}{[f^{'}(x)]^2}, \therefore  \phi^{'}(x) =1-\frac{1}{m}<1$

有重根时Newton迭代法是收敛的,且为一阶收敛:$\lim_{k\rightarrow \infty}\frac{|e_{k+1}|}{e_k}=1-\frac{1}{m}$。

\paragraph{Newton迭代法加速}
\begin{itemize}
    \item $\phi_1(x)=x-m\frac{f(x)}{f^{'}(x)}$
    \item $\phi_2(x)=x-\frac{[f^{'}(x)]^2}{[f^{'}(x)]^2-f(x)f^{{'}{'}}(x)}$
\end{itemize}

\subsection{割线法}

将\ref{eq:4.1}的$f^{'}(x_k) $用其近似差商代替,i.e.$f^{'}(x_k)\approx \frac{f(x_k)-f(x_{k-1})}{x_k-x_{k-1}}$将此代入Newton法即可得:
\begin{equation}
    x_{k+1}=x_k-\frac{f(x_k)(x_k-x_{k-1})}{f(x_k)-f(x_{k-1})}
    \label{eq:4.2}
\end{equation}
\begin{theorem}
    设$f(x)$在其零点$x^*$邻域二阶连续可导,且$f^{'}(x^*)\neq 0$,则$\exists \varepsilon >0$,当$x_0, x_1\in [x^*-\varepsilon, x^*+\varepsilon]$时,割线法\ref{eq:4.2}收敛,且收敛阶为$p=\frac{1+\sqrt{5}}{2}\approx 1.618$
\end{theorem}

\section{非线性方程组的不动点迭代方法}
设有如下非线性方程组:
\begin{equation}
    f_1(x_1,x_2,\dots, x_n)=0,f_2(x_1,x_2,\dots, x_n)=0,\dots, f_n(x_1, x_2, \dots, x_n)=0
    \label{eq:5.1}
\end{equation}
令$x=(x_1, x_2, \dots, x_n)^T, F(x)=(f_1, f_2, \dots, f_n)^T$则\ref{eq:5.1}可写成如下形式:
\begin{equation}
    F(x)=0\\
    \label{eq:5.2}
\end{equation}

\subsection{基本概念}
假设$f_1, f_2, \dots, f_n$是定义在子集$D\subset R^n$上的实值函数,则可得$F:D\subset R^n\rightarrow R^n$表示
F为$R^n$中子集D到$R^n$的映射。更一般的,$F:D\subset R^n\rightarrow R^m$,若$F(x)$为标量,则$\lim_{\Delta x\rightarrow 0}\frac{F(x+\Delta x)-F(x)-F^{'}(x)\Delta x}{\Delta x}=0 $

\begin{definition}
    设$F:D\subset R^n\rightarrow R^m$,若$\exists A\in R^{m\times n} $,对D内x及$\Delta x\in R^n$有:
    $$\lim_{\Delta x\rightarrow 0}\frac{||F(x+\Delta x)-F(x)-A\Delta x||}{||\Delta x||}=0$$
    则称F在x可导,称A为F在x的Frehet导数,记为
    $$F^{'}(x)=A=\left[
        \begin{matrix}
            \frac{\partial f_1}{\partial x_1} & \frac{\partial f_1}{\partial x_2} & \dots & \frac{\partial f_1}{\partial x_n} \\
            \vdots & \vdots & \dots & \vdots \\
            \frac{\partial f_n}{\partial x_1} & \frac{\partial f_n}{\partial x_2} & \dots & \frac{\partial f_n}{\partial x_n}
        \end{matrix}
        \right] $$
\end{definition}

\begin{theorem}
    设$F:D\subset R^n\rightarrow R^m$在$x\in D\subset R^n$是Frechet可微的,则$F(x)$在x连续,i.e.$\lim_{\Delta x\rightarrow 0}||F(x+\Delta x)-F(x)||=0$
\end{theorem}

\subsection{不动点迭代法}
将$F(x)=0$转化成等价的方程组
\begin{equation}
    x=\phi(x)
    \label{eq:5.3}
\end{equation}
其中$\phi D\subset R^n\rightarrow R^n$,且$\phi$是连续的,若$\exists x^*\in D$使得$x^*=\phi(x^*)$,则称$x^*$为$\phi(x)$的不动点,由此可构造如下迭代方法
\begin{equation}
    x^{(k+1)}=\phi(x^{(k)}), k=0,1,2,\dots
    \label{eq:5.4}
\end{equation}
其中$x^{(k)}=(x^{(k)}_1, x^{(k)}_2,\dots, x^{(k)}_n) $

\begin{definition}
    设$\{x^{(k)} \}$收敛于$x^*$,
    \begin{enumerate}
        \item 若$\exists$实数$p\ge 1$及常数$C>0$,使
        \begin{equation*}
            \lim_{k\rightarrow \infty}\frac{||x^{(k+1)}-x^*||}{||x^{(k)}-x^*||^p}=C
        \end{equation*}则称$\{x^{(k)} \}$为p阶收敛的;其中$p=1$时,$0<C<1$
        \item 若存在实数$p\ge 1$和常数$C>0$,使$||x^{(k+1)}-x^* ||\le C||x^{(k)}-x^* ||^p, k=0,1,2,\dots$成立,其中$p=1$时,$0<c<1$,则称$\{x^{(k)} \}$至少p阶收敛;当p=1时,$\{x^{(k)} \}$至少线性收敛
        \item 若存在实数$p\ge 1$,且$\lim_{k\rightarrow 0}\frac{||x^{(k+1)}-x^* ||}{||x^{(k)}-x^* ||^p}=0 $,则称$\{x^{(k)} \}$为超p阶收敛。
    \end{enumerate}
\end{definition}

\begin{theorem}
    \textbf{(压缩映射原理)}设$D\subset R^n$中的一个闭集,若$\phi:D\rightarrow D (i.e. \phi(x)\in D, \forall x \in D)$为压缩映射,即满足如下条件:
    \begin{equation}
        ||\phi(x)-\phi(y) ||\le L||x-y||, 0<L<1, \forall x,y\in D
    \end{equation}
    则有如下结论:
    \begin{enumerate}
        \item $\forall x^{(0)}\subset D $,由\ref{eq:5.4}产生的迭代序列$\{x^{(k)} \}\subset D$
        \item $\phi$在D上存在唯一的不动点$x^*$使$x^*=\phi(x^*)$
        \item $||x^{(k+1)}-x^* ||\le ||x^{(k)}-x^* ||,k=0,1,2,\dots$,i.e.$\{x^{(k)} \}$至少线性收敛
        \item 有误差估计式:
        \begin{equation}
            ||x^{(k)}-x^* ||\le \frac{L^k}{1-L}||x^{(1)}-x^{(0)} ||
        \end{equation}
    \end{enumerate}
\end{theorem}
\begin{proof}
    \begin{enumerate}
        \item 由映内性,显然
        \item 由压缩性的$||x^{(k+1)}-x^{(k)}||=||\phi(x^{(k)})-\phi(x^{(k-1)})||\le L||x^{(k)}-x^{(k-1)}||\le \dots \le L^k||x^{(1)}-x^{(0)} ||$\\
        于是对$\forall m \in Z$,有
        \begin{equation*}
            \begin{split}
                ||x^{(k+m)}-x^{(k)}|| &= ||(x^{(k+m)}-x^{(k+m-1)})+(x^{(k+m-1)}-x^{(k+m-2)})+\\
                &\quad \dots + (x^{(k+1)}-x^{(k)}) ||\\
                &\le (L^{k+m-1}+L^{k+m-2}+\dots+L^k)||x^{(1)}-x^{0}|| \\
                &\le \frac{L^k}{1-L}||x^{(1)}-x^{0}||
            \end{split}
        \end{equation*}
        $\because 0<L<1$,当$k\rightarrow\infty$时,$||x^{(k+m)}-x^{(k)}||\rightarrow 0$\\
        $\because \{x^{(k)} \}$为一个柯西序列,$\{x^{(k)} \}$有一个有限极限,$\lim_{k\rightarrow\infty}x^{(k)}=x^* $\\
        又根据定理条件,$\phi(x)$在D上连续$\therefore \lim_{k\rightarrow \infty}x^{(k+1)}=\lim_{k\rightarrow \infty}\phi(x^{(k)})=\phi(x^*), \lim_{k\rightarrow \infty}x^{(k+1)}=x^* $\\
        $\therefore \phi(x^*)=x^* $\\
        $\because$D为闭集,且$x^{(k)}\in D, \forall k=0,1,2,\dots \therefore \{x^{(k)} \}$的极限$x^*\in D$

        假设还有另一个不动点$y^*\in D, x^*\neq y^*$,则$||x^*-y^*||=||\phi(x^*)-\phi(y^*) ||\le L||x^*-y^*||<||x^*-y^*||$,矛盾。
        \item $||x^{(k+1)}-x^*||=||\phi(x^{(k)})-\phi(x^*) ||\le L||x^{(k)}-x^*||$,且$0<L<1$,至少线性收敛
        \item $||x^{(k+m)}-x^{(k)}||\le \frac{L^k}{1-L}||x^{(1)}-x^{(0)}||$\\
            当$m\rightarrow\infty$时,$||x^{(k)}-x^*||\le \frac{L^k}{1-L}||x^{(1)}-x^{(0)}||$
    \end{enumerate}
\end{proof}

\begin{definition}
    设$x^*$为$\phi$的不动点,若$\exists x^*$的一个邻域$S(x^*)\subset D, \forall x^{(0)}\in S(x^*) $,有\ref{eq:5.4}产生的序列$\{x^{(k)} \}\subset S(x^*) $,且$\lim_{k\rightarrow\infty}x^{(k)}=x^* $则称$\{x^{(k)} \}$有局部收敛性。
\end{definition}

\begin{theorem}
    \label{theorem:5.3}
    设$\phi:D\subset R^n \rightarrow R^n, x^*\in D$为$\phi$的不动点,若$\exists$一个开球,$S_r(x^*)=\{x|||x-x^*||<r, r>0\}\subset D $,使得
    $||\phi(x)-\phi(x^*) ||\le L||x-x^*||,\forall x\in S_r(x^*), 0\le L<1 $则对$\forall x^{(0)}\in S_r(x^*) $,由\ref{eq:5.4}产生的$\{x^{(k)} \}$具有如下性质:
    \begin{enumerate}
        \item $x^{(k)}\in S_r(x^*), k=0,1,2,\dots $
        \item $\lim_{k\rightarrow\infty}x^{(k)}=x^* $
        \item $\{x^{(k)} \}$至少线性收敛。
    \end{enumerate}
\end{theorem}
\begin{proof}
    \begin{enumerate}
        \item 数学归纳法证明 $||x^{(k+1)}-x^* ||<r$
        \item $||x^{(k+1)}-x^* ||=||\phi(x^{(k)})-\phi(x^*) ||\le L||x^{(k)}-x^*||\le \dots \le L^{k+1}||x^{(0)}-x^* || \rightarrow 0, k\rightarrow\infty$
        \item $||x^{(k+1)}-x^* ||\le L||x^{(k)}-x^*||, 0<L<1$至少线性收敛
    \end{enumerate}
\end{proof}

\begin{theorem}
    设$\phi:D\subset R^n \rightarrow R^n, \phi$在D内有不动点$x^*$,且$\phi$在$x^*$可导,$\rho(\phi^{'}(x^*))=\lambda <1$,则$\exists$开球$S_r(x^*)\subset D$对$\forall x^{(0)}\in S_R(x^*) $,由\ref{eq:5.4}产生的迭代序列$\{x^{(k)} \}$收敛到$x^*$
\end{theorem}
\begin{proof}
    $\because \lambda<1, \therefore \exists \varepsilon >0, s.t. \lambda+2\varepsilon<1$\\
    由范数与谱半径的关系可知,对于$\varepsilon$,存在一种从属范数$||\cdot||_\varepsilon$,使得$||\phi(x^*) ||_\varepsilon<\lambda+\varepsilon$。又$\because \phi(x)$在$x^*$可导,$\therefore $由导数定义可知,对于$\varepsilon>0, \exists S_r(x^*), \forall x\in S_r(x^*), ||\phi(x)-\phi(x^*)-\phi^{'}(x^*)(x-x^*) ||_\varepsilon\le \varepsilon||x-x^*||_\varepsilon $
    \begin{equation*}
        \begin{split}
            ||\phi(x)-\phi(x^*) ||_\varepsilon &=||\phi(x)-\phi(x^*)-\phi^{'}(x^*)(x-x^*)+\phi^{'}(x^*)(x-x^*) ||_\varepsilon\\
            &\le ||\phi(x)-\phi(x^*)-\phi^{'}(x^*)(x-x^*)||_\varepsilon+||\phi^{'}(x^*)(x-x^*) ||_\varepsilon\\
            &< \varepsilon||x-x^*||_\varepsilon + (\lambda+\varepsilon)||x-x^*||_\varepsilon\\
            &= (\lambda+2\varepsilon)||x-x^*||_\varepsilon
        \end{split}
    \end{equation*}
    $\because \lambda+2\varepsilon <1, \therefore$由定理\ref{theorem:5.3}可知,$\{x^{(k)} \}$收敛于$x^*$
\end{proof}

\begin{example}
    用不动点迭代法求解方程组
    $$\left\{\begin{array}{lr}
        x_1^2-10x_1+x_2^2+8=0\\
        x_1x_2^2+x_1-10x_2+8=0
    \end{array}\right.$$
    
    解:将原方程组转化为$x=\phi(x)$的形式,其中$x=\left[\begin{matrix}
        x_1\\x_2
    \end{matrix}\right], \phi(x)=\left[\begin{matrix}
        \phi_1(x)\\ \phi_2(x)
    \end{matrix}\right]=\left[\begin{matrix}
        \frac{x_1^2+x_2^2+8}{10} \\ \frac{x_1x_2^2+x_1+8}{10}
    \end{matrix}\right], \therefore x^{(k+1)}=\phi(x^{(k)}), k=0,1,2,\dots $

    设$D=\{(x_1,x_2)|0\le x_1, x_2\le 1.5 \}$易证$0.8\le \phi_1(x)\le 1.25, 0.8\le \phi_2(x)\le 1.2875 \therefore \phi(x)\in D $(映内性)\\
    $\forall x, y \in D$,有
    \begin{equation*}
        \left\{
            \begin{array}{lr}
                |\phi_1(x)-\phi_1(y)|=\frac{|y_1^2-x_1^2+y_2^2-x_2^2|}{10}\le \frac{3}{10}(|y_1-x_1|+|y_2-x_2|)=\frac{3}{10}||x-y||_1 \\
                |\phi_2(x)-\phi_2(y)|=\frac{|y_1y_2^2-x_1x_2^2+y_1-x_1|}{10}\le \frac{4.5}{10}||x-y||_1
            \end{array}
            \right.
    \end{equation*}
    $\therefore ||\phi(x)-\phi(y)||=||\phi_1(x)-\phi_1(y)||+||\phi_2(x)-\phi_2(y)||\le \frac{7.5}{10}||x-y||_1$(压缩性)\\
    $\therefore$迭代序列收敛。

    取$x^{(0)}=(0,0)^T $,则$x^{(1)}=(0.8, 0.8)^T, x^{(2)}=(0.928, 0.9312)^T, \dots x^{(6)}=(0.99938, 0.99932)^T, x^*=(1,1)^T$
    $$\phi^{'}(x)= \left[\begin{matrix}
        \frac{1}{5}x_1 & \frac{1}{5} x_2 \\
        \frac{x_2^2+1}{10} & \frac{x_1x_2}{5}
    \end{matrix}\right], \therefore \phi^{'}(x^*)=\left[\begin{matrix}
        0.2 & 0.2 \\
        0.2 & 0.2
    \end{matrix}\right] $$
    $\therefore ||\phi^{'}(x^*) ||_1=0.4<1 \therefore \rho(\phi^{'}(x^*)\le ||\phi^{'}(x^*)||_1<1 )$\\
    $\therefore \{x^{(k)} \}$收敛
\end{example}

\subsection{Newton迭代法}
$x_{k+1}=x_k-\frac{f(x)}{f^{'}(x)} $推广到方程组:
\begin{equation}
    x^{(k+1)}=x^{(k)}-[F^{'}(x^{(k)})]^{-1}F(x^{(k)}) , k=0,1,2,\dots
    \label{eq:5.7}
\end{equation}
将\ref{eq:5.7}改写成如下形式:
\begin{equation*}
    \begin{array}{lr}
        F^{'}(x^{(k)})(x^{(k+1)}-x^{(k)})=-F(x^{(k)})\\
        F^{'}(x^{(k)})\Delta x^{(k)}=-F(x^{(k)}) \\
        x^{(k+1)}=x^{(k)}+\Delta x^{(k)}
    \end{array}
    ,k=0,1,2,\dots
\end{equation*}

\begin{example}
    $$\left\{
        \begin{array}{lr}
            x_1^2-10x_1+x_2^2+8=0\\
            x_1x_2^2+x_1-10x_2+8=0
        \end{array}\right.
        $$
    
    解:
    $$F^{'}(x)=\left[\begin{matrix}
        2x_1-10 & 2x_2 \\
        x_2^2+1 & 2x_1x_2-10
    \end{matrix}\right] $$
    取$x^{(0)}=(0, 0)^T $由$F^{'}(x^{(k)})\Delta x^{(0)}=-F(x^{(0)})\Rightarrow \Delta x^{(0)}=\left[\begin{matrix}
        0.8 \\ 0.88
    \end{matrix}\right] $\\
    $\therefore x^{(1)}=x^{(0)}+\Delta x^{(0)}=(0.8, 0.88)^T, \dots, x^{(4)}=(1.000, 1.000)^T $
\end{example}

\begin{theorem}
    设$F:D\subset R^n\rightarrow R^n, F(x^*)=0, F(x)$在$x^*$的开邻域$S_0\subset D$上可导,且$F^{'}(x) $连续,$[F^{'}(x^*)]^{-1}\exists$,则:
    \begin{enumerate}
        \item 存在闭球$\bar{S_r^{'}}=\{x|||x-x^*||\le r, r>0\}\subset S_0, \forall x^{(0)}\subset \bar{S_r^{'}} $,
            由$x^{(k+1)}=x^{(k)}-[F^{'}(x^{(k)})]^{-1}F(x^{(k)}) $产生的$x^{(k)}\in \bar{S_r^{'}}, k=0,1,2,\dots$(数学归纳法证明)
        \item $\lim_{k\rightarrow\infty} x^{(k)}=x^*$(局部收敛性,$\rho(\phi^{'}(x^*))<1$或某种从属范数<1)
        \item 若存在常数$k>0, s.t. ||F^{'}(x)-F^{'}(x^*) ||\le K||x-x^*||, \forall x \in \bar{S_r^{'}}$,则$\{x^{(k)} \}$至少平方收敛。($||x^{(k+1)}-x^*||\le C||x^{(k)}-x^*||^2$)
    \end{enumerate}
\end{theorem}

\subsection{拟Newton法与Broyden方法}
将\ref{eq:5.7}中的$[F^{'}(x^{(k)})]^{-1} $用$A_k^{-1} $代替,则
\begin{equation}
    x^{(k+1)}=x^{(k)}-A_k^{-1}F(x^{(k)}), k=0,1,2
    \label{eq:5.8}
\end{equation}
$f^{'}(x^{(k)})(x_k-x_{k-1})\approx f(x_k)-f(x_{k-1}) $,由此分析得:
\begin{align}
    &A_{k+1}(x^{(k+1)}-x^{(k)})=F(x^{(k+1)})-F(x^{(k)}) \label{eq:5.9}\\
    &A_{k+1} = A_k+\Delta A_k, rank(\Delta A_k)=m\ge 1 \label{eq:5.10}
\end{align}

由(\ref{eq:5.8})(\ref{eq:5.9})(\ref{eq:5.10})组成的迭代法称为拟Newton方法。

若$m=1$,则
\begin{equation}
    \Delta A_k=u^{(k)}[v^{(k)}]^T, u^{(k)},v^{(k)}\in R^n 
    \label{eq:5.11}
\end{equation}
记$S^{(k)}=x^{(k+1)}-x^{(k)}, y^{(k)}=F(x^{(k+1)})-F(x^{(k)}) $,则\ref{eq:5.9}可写成
\begin{equation}
    A_{k+1}S^{(k)}=y^{(k)} 
    \label{eq:5.12}
\end{equation}
由(\ref{eq:5.11})(\ref{eq:5.12})得
\begin{equation*}
    \begin{array}{lr}
        \{A_k+u^{(k)}[v^{(k)}]^T \}S^{(k)}=y^{(k)} \Rightarrow \\
        u^{(k)}[v^{(k)}]^TS^{(k)} = y^{(k)}-A_kS^{(k)} \Rightarrow \\
        u^{(k)} = \frac{1}{[v^{(k)}]^TS^{(k)}}(y^{(k)}-A_kS^{(k)}), [v^{(k)}]^TS^{(k)}\neq 0
    \end{array}
\end{equation*}
将$u^{(k)}$代入$\Delta A_k$有:
$$\Delta A_k = \frac{y^{(k)}-A_kS^{(k)}}{[v^{(k)}]^TS^{(k)}}[v^{(k)}]^T$$
选取$v^{(k)}=S^{(k)} $,则有$\Delta A_k = \frac{y^{(k)}-A_kS^{(k)}}{[S^{(k)}]^TS^{(k)}}[S^{(k)}]^T$

$A_0=I, A_{k+1}=A_k+\Delta A_k $由此可得秩1拟Newton法如下:
\begin{align}
    x^{(k+1)}&=x^{(k)}-A^{-1}_kF(x^{(k)}) \\
    A_{k+1}&=A_k + (y^{(k)}-A_kS^{(k)})\frac{[S^{(k)}]^T}{[S^{(k)}]^TS^{(k)}}
\end{align}


\end{document}