%!TEX program=xelatex
\documentclass[a4paper]{article}

\usepackage{amsmath}
\usepackage{amsthm}
\newtheorem{definition}{定义}[section]
\newtheorem{theorem}{定理}[section]
\newtheorem{lemma}{引理}[section]
\newtheorem{example}{例}[section]

\makeatletter % `@' now normal "letter"
\@addtoreset{equation}{section}
\makeatother  % `@' is restored as "non-letter"
\renewcommand\theequation{\oldstylenums{\thesection}%
                   .\oldstylenums{\arabic{equation}}}

\DeclareMathOperator{\spn}{span}

\usepackage{ctex}
\usepackage{extarrows}
\usepackage{amssymb}
\usepackage{tabularx}
\usepackage{multicol}
\usepackage{multirow}
\usepackage{booktabs}
\usepackage{hyperref}

\title{第二章\ 线性代数方程组的数值解法}
\author{}
\date{}

\begin{document}
\maketitle

\section{引言和线性代数基础知识}

\begin{example}
$Ax=b \Rightarrow x=A^{-1}b$ \\
高斯消元法复杂度太高$O(n^3)$,在$A$规模很大时,时间消耗过于巨大。
\end{example}
\subsection{线性空间}
\begin{definition}
定义了加减和数乘的非空集合,称为线性空间。
\end{definition}
如$R^n, C^n, R^{n\times m}(C^{n\times m}), C[a,b] \text{(定义在[a,b]上的连续函数)}$。

\subsection{内积}
\begin{definition}
对于定义于数域$K(R \ or\  C)$上的线性空间$X$,如果对于$\forall$$u, v, w \in X$及$\alpha \in K$,满足:
\begin{itemize}
\item $(u+v, w)=(u, w) + (v, w)$
\item $(\alpha u, v)=\alpha (u, v)$
\item $(u, v)=\overline{(v, u)}$
\item $(u, u)\ge 0 且 (u, u)=0 \Leftrightarrow u=0$
\end{itemize}
则称数$(u,v)$为$u$与$v$的内积,定义了内积的线性空间称为内积空间。
\end{definition}

几种常见内积的定义:
\begin{itemize}
\item $R^n$和$C^n$上的内积:\\
设$x,y \in R^n$,则$R^n$的内积可定义为:\\
$$(x,y)=\sum^{n}_{i=1}x_iy_i$$\\
若给定权系数$w_i(i=1,2,\dots,n)$,则$R^n$上带权$\{w_i\}$的内积可定义为:\\
$$(x,y)=\sum^{n}_{i=1}w_ix_iy_i$$\\
类似的,若$x,y\in C^n$,则带权的内积:\\
$$(x,y)=\sum^{n}_{i=1}w_ix_i\overline{y_i}$$
\item C[a,b]上的内积:
\begin{definition}
若定义在$[a,b]$上的函数$\rho (x)$满足:
\begin{itemize}
\item $\rho(x) \ge 0, \forall x \in (a,b)$
\item $\int^{b}_{a}x^k\rho(x)dx, \exists$ 且有限$(k=0,1,2,\dots)$
\item 若对$[a,b]$上的非负连续函数$g(x)$有:\\
$\int^{b}_{a}\rho(x)g(x)dx=0$,则$g(x)\equiv 0$
\end{itemize}
就称$\rho(x)$为$[a,b]$上的一个权函数。
\end{definition}
设$f,g\in C[a,b]$,$\rho$是$[a,b]$上给定的权函数,则称:\\
$(f,g)=\int^{b}_{a}\rho(x)f(x)g(x)dx$为$C[a,b]$上函数$f,g$的内积。
\end{itemize}

\subsection{内积空间的几个性质}
\begin{itemize}
\item (Cauchy-Schwarz不等式):设$X$为一个内积空间,则对$\forall u,v\in X$有:
\begin{equation}
|(u,v)|^2\le (u,u)\cdot (v,v)
\end{equation}
\item 设$X$为一个内积空间,$u_1,u_2,\dots,u_n\in X$,矩阵:
$$G=\left[
 \begin{matrix}
   (u_1,u_1) & (u_2, u_1) & \dots &(u_n, u_1) \\
   (u_1,u_2) & (u_2, u_2) & \dots&(u_n,u_2) \\
   \dots & \dots & \dots & \dots \\
   (u_1, u_n) & (u_2, u_n) & \dots& (u_n, u_n)
  \end{matrix}
  \right] $$
则G非奇异的充要条件是$u_1,u_2,\dots,u_n$线性无关
\item (Gram-Schmidz正交化方法):若$\{u_1,u_2,\dots,u_n\}$是内积空间$X$中一个线性无关的序列,则可按下列公式:
\begin{equation}
\left\{  
	\begin{array}{lr}  
	v_1 = u_1, &  \\  
	v_i = u_i - \sum^{i-1}_{k=1}\frac{(u_i,v_k)}{(v_k,v_k)}v_k, & i=2,3,\dots, n 
	\end{array}  
\right.  
\end{equation}
构造一个正交序列$\{v_1, v_2, \dots, v_n\}$且满足:$(v_i,v_j)=0,(i\neq j)$。
\end{itemize}

\subsection{向量范数}
\begin{example}
欧式范数或2-范数:\\
$$||\mathbf{x}||_2=\sqrt{(\mathbf{x},\mathbf{x})}=\sqrt{\sum^n_{i=1}x_i^2}$$
\end{example}

\begin{definition} 
设对$ \forall \mathbf{x} \in R^n(or \ C^n)$,按一定的规则有一实值函数与之对应,记为$||\mathbf{x}||$,若满足:
\begin{itemize}
\item 正定性:$||\mathbf{x}|| \ge 0$ 且 $||\mathbf{x}||=0 \Leftrightarrow x=0$
\item 齐次性:$||\alpha \mathbf{x}||=|\alpha|\cdot ||\mathbf{x}||, \forall \alpha \in R (or \ C)$
\item 三角不等式 
\begin{equation}
\label{eqtri}
||\mathbf{x}+\mathbf{y}||\le ||\mathbf{x}||+||\mathbf{y}||, \forall \mathbf{x},\mathbf{y} \in R^n(or \ C^n)
\end{equation}
\end{itemize}
则称$||\mathbf{x}||$为向量$\mathbf{x}$的范数,由(\ref{eqtri})易证$|||\mathbf{x}||-||\mathbf{y}|||\le ||\mathbf{x}-\mathbf{y}||$ (自证)
\end{definition}

\begin{proof}
范数三角不等式减法形式:\\
不妨设:$||\mathbf{x}|| \ge ||\mathbf{y}||$,则:
\begin{equation*}
\begin{split}
& |||\mathbf{x}||-||\mathbf{y}||| \\
=& ||\mathbf{x}|| - ||\mathbf{y}|| \\
=& ||\mathbf{x} - \mathbf{y} + \mathbf{y}|| - ||\mathbf{y}|| \\
\le& ||\mathbf{x} - \mathbf{y}|| + ||\mathbf{y}|| - ||\mathbf{y}|| \\
=& ||\mathbf{x} - \mathbf{y}||
\end{split}
\end{equation*}
\end{proof}


几种常见范数:
\begin{itemize}
\item $||\mathbf{x}||_\infty = \max \lim_{1\le i \le n}|\mathbf{x}_i|$, $\infty$-范数
\item $||\mathbf{x}||_1=\sum^{n}_{i=1}|x_i|$,1-范数
\item $||\mathbf{x}||_2=(\sum^n_{i=1}x_i^2)^\frac{1}{2}=\sqrt{(x,x)}$,2-范数
\end{itemize}


\begin{proof}
二范数三角不等式:\\
\begin{equation*}
\begin{split}
&||\mathbf{x}+\mathbf{y}||_2 \\
=& (\mathbf{x}+\mathbf{y}, \mathbf{x}+\mathbf{y})^{\frac{1}{2}} \\
=&[(\mathbf{x},\mathbf{x}) + 2(\mathbf{x}, \mathbf{y}) + (\mathbf{y}, \mathbf{y})]^{\frac{1}{2}} \\
\le& [||\mathbf{x}||^2_2 + 2||\mathbf{x}||_2\cdot||\mathbf{y}||_2 + ||\mathbf{y}||_2^2]^{\frac{1}{2}} \\
=& ||\mathbf{x}||_2+||\mathbf{y}||_2\\
\end{split}
\end{equation*}
\end{proof}

\begin{example}
思考题:对于任一种范数$||\cdot||_\lambda$定义集合:
$$A=\{x|x\in R^3, ||x|| \le 1\}$$
试问当$\lambda=1,2,\infty$时,A代表什么样的图形?并画图
\end{example}

\begin{equation}
||x||_p = (\sum_{i=1}|x_i|^p)^{\frac{1}{p}}, p \in [1, \infty)
\end{equation}

\subsection{向量范数的性质}
\begin{itemize}
\item 连续性:
\begin{theorem}
设给定$A\in R^{n \times n}$,则对$R^n$上的每一种向量范数$||\cdot||, x\in R^n$,$||Ax||$都是x的分量$(x_1, x_2, \dots, x_n)$的n元连续函数$f(x_1, x_2, \dots, x_n)$
\label{theorem_con}
\end{theorem}
\item 等价性:
\begin{definition}
线性空间$X$上定义了两种范数$||\cdot||_\alpha$和$||\cdot||_\beta$,若$\exists$常数$C_1, C_2 > 0$,使得
$$C_1\cdot ||\mathbf{u}||_\alpha \le ||\mathbf{u}||_\beta \le C_2\cdot ||\mathbf{u}||_\alpha, \forall \mathbf{u} \in X$$
则称$||\cdot||_\alpha$和$||\cdot||_\beta$是$X$上等价的范数。
\label{def:equal}
\end{definition}
\item 传递性:\\
若$||\cdot||_\alpha$与$||\cdot||_\beta$等价,$||\cdot||_\beta$与$||\cdot||_\lambda$等价,则$||\cdot||_\alpha$与$||\cdot||_\lambda$等价。
\begin{theorem}
$R^n$上所有范数是彼此等价的。
\end{theorem}
\end{itemize}

\subsection{矩阵范数}
\begin{definition}
如果对$R^{n\times n}$上的任一矩阵A,对应一个实数$||A||$,满足条件:$\forall A,B \in R^{n\times n}$和$\alpha \in R$,
\begin{itemize}
\item $||A|| > 0$,且$||A||=0 \Leftrightarrow A=0$\hfill(正定性)
\item $||\alpha A||=|\alpha|\cdot||A||$\hfill (齐次性)
\item $||A+B|| \le ||A|| + ||B||$\hfill(三角不等式)
\item $||AB||\le ||A||\cdot||B||$\hfill (相容性)
\end{itemize}
则称$||A||$为矩阵A的范数\\
\end{definition}
易证:$|||A||-||B|||\le ||A-B||$(自证)

\paragraph{F-范数}
$$||A||_F=[\sum^n_{i=1}\sum^n_{j=1}|a_{ij}|^2]^\frac{1}{2}$$
性质4利用柯西不等式证明(和的平方$\le$平方的和)

\begin{definition}
对于$R^{n\times n}$上给定的一种矩阵范数:\\
$||\cdot||_{\beta}$,$R^n$上规定的一种向量范数$||\cdot||_\alpha $,若有:
\begin{equation}
||A\mathbf{x} ||_\alpha \le ||A||_\beta ||\mathbf{x} ||_\alpha, \forall A \in R^{n\times n}, \mathbf{x} \in R^{n}
\end{equation}
成立,则称上述矩阵范数和向量范数是相容的。
或:
\begin{equation}
||A||_\beta \xlongequal[]{\sup} \frac{||A\mathbf{x} ||_\alpha}{||\mathbf{x}||_\alpha}
\end{equation}
\end{definition}

\begin{theorem}
设$A\in R^{n\times n}$,$||\cdot||$是$R^n$上的向量范数,则:
\begin{equation}
||A||=\max_{||x||=1}||Ax||=\max_{x \neq 0}\frac{||Ax||}{||x||}
\label{eq_natnorm}
\end{equation}
是一种矩阵范数,称其为由向量范数诱导出的矩阵范数(简称为诱导范数,或从属范数,自然范数),且满足相容性条件。
\label{theorem_natnorm}
\end{theorem}

\begin{proof}
\begin{itemize}
\item 由定理\ref{theorem_con}可知,$||Ax||$是$R^n$中有界闭集$D=\{\mathbf{x}=(x_1, x_2, \dots, x_n)^T, ||x||=1\}$上的连续函数,$\therefore ||Ax||$在D上最大值存在,$\therefore$(\ref{eq_natnorm})中$\max_{||x||=1}||Ax||$的写法正确。
\item 对于$\forall x \neq 0$,
$$\because \frac{||Ax||}{||x||}=||A\cdot \frac{x}{||x||}||,\text{且}||\frac{x}{||x||}||=1$$
$$\therefore \max_{x \neq 0} \frac{||Ax||}{||x||} = \max_{x \neq 0 }||A\cdot \frac{x}{||x||}|| \xlongequal[]{let y = \frac{x}{||x||}} \max_{||y||=1}||Ay|| \xlongequal[]{\text{记y为x}}\max_{||x||=1}||Ax||$$
\item 
\begin{itemize}
\item 正定性:显然$||A|| \ge 0$,另外:\\
若$A=0$,则:$||A||=\max_{||x||=1}||Ax||=0$,反之,若$||A||=0$,i.e. \ 对$\forall $非零向量x有:
$$||Ax||=0\Rightarrow Ax = 0 \Rightarrow A=0$$
\item 齐次性:对$\forall \alpha \in R$,有:
\begin{equation*}
\begin{split}
&||\alpha A||\\
=& \max_{||x||=1}||\alpha Ax||\\
=&\max_{||x||=1}|\alpha|||Ax|| \\
=&|\alpha|\max_{||x||=1}||Ax|| \\
=&|\alpha|\cdot||A||
\end{split}
\end{equation*}
\item 三角不等式: $\forall A, B \in R^{n\times n}$,有:
\begin{equation*}
\begin{split}
&||A + B||\\
=& \max_{||x||=1}||(A+B)x||\\
=&\max_{||x||=1}||Ax+Bx|| \\
\le&\max_{||x||=1}(||Ax||+||Bx||) \\
\le& ||A|| + ||B||
\end{split}
\end{equation*}
\item 相容性: 由\ref{eq_natnorm}显然有$||Ax||\le ||A||||x||, \forall x \in R^n$,\\
$\therefore A, B \in R^{n\times n}, \text{有}$ \\
\begin{equation*}
\begin{split}
&||AB||\\
=& \max_{||x||=1}||ABx||\\
=&\max_{||x||=1}||A(Bx)|| \\
\le&\max_{||x||=1}||A||\cdot||Bx||\\
\le& ||A||\max_{||x||=1}||Bx|| \\
=& ||A||\cdot ||B||
\end{split}
\end{equation*}
\end{itemize}
\end{itemize}
综上所述,$||A||$是$R^{n\times n}$上的一种矩阵范数。\\
由定义及$||A||$是$R^{n\times n}$是一种范数,即可知其相容性条件成立。
\end{proof}

注释:
\begin{itemize}
\item 对$\forall $的从属范数,$||I|| = 1$
\item 矩阵的任一从属范数一定与所给定的向量范数相容,但相容未必具有从属关系。如:$||Ax||_2 \le ||A||_F\cdot |||x||_2$,但不具有从属关系。\\
事实上,$||I||_F = \sqrt{n}$,另外,$||I||_F=\max_{x \neq 0 }\frac{||Ix||}{||x||} = 1$ \\
$\therefore$$||A||_F$与$||x||_2$没有关系。
\end{itemize}

\begin{theorem}
设$x \in R^n, A \in R^{n\times n}$,则:
\begin{itemize}
\item $||A||_\infty = \max_{||x||_\infty = 1}||Ax||_\infty = \max_{1\le i \le n}\sum_{j=1}^{n}|a_{ij}|$ \hfill ($\infty$-范数 or 行范数)
\item $||A||_1 = \max_{||x||_1=1}||Ax||_1=\max_{1 \le j \le n}\sum^n_{i=1}|a_{ij}|$ \hfill (1-范数 or 列范数)
\item $||A||_2=\max_{||x||_2 = 1}||Ax||_2=\sqrt{\lambda_{\max(A^TA)}}$ \hfill (2-范数 or 谱范数)\\
其中,$\lambda_{\max(A^TA)}=\rho(A^TA)$($\rho$表示谱半径)表示$A^TA$的最大特征值。\\
一般的,$\rho(A)=\max_{1 \le i \le n}|\lambda_i|$,$\lambda_i$为A的特征值。
\end{itemize}
\end{theorem}

\begin{proof}
\begin{itemize}
\item 若$A=0$时,显然成立,不妨设$A \neq 0 $
\\$\forall x \in R^n$,且$||x||_\infty = 1$,则:
$$||Ax||_\infty = \max_{1 \le i \le n}|\sum^n_{j=1}a_{ij}x_j| \le \max_i\sum^n_{j=1}|a_{ij}||x_j|\le \max_i\sum^n_{j=1}|a_{ij}|\xlongequal[]{\text{记}} \mu$$
$\therefore \max_{||x||_\infty=1}||Ax||_\infty \le \mu$\\
事实上,假设第$i_0$行使$\mu = \sum_{j=1}|a_{i_0j}|$成立,\\
取$x^{(0)}=(x_1^{(0)}, x_2^{(0)}, x_3^{(0)}, \dots, x_n^{(0)})^T$,其中:
$$ x_j^{(0)}=\left\{
 \begin{array}{lr} 
 1, a_{i_0j} \ge 0 \\
 -1, a_{i_0j} < 0   \\         
 \end{array}
  \right. j=1, 2, \dots, n$$
显然$||x^{(0)}||_\infty=1$且\\
$$\max_{||x||_\infty=1}||Ax||_\infty \ge ||Ax^{(0)}||_\infty = \max_i|\sum^n_{j=1}a_{ij}x^{(0)}_j|\ge|\sum^n_{j=1}a_{i_0j}x^{(0)}_j|=\sum^n_{j=1}|a_{i_0j}|=\mu$$
$\therefore \max_{||x||_\infty=1}||Ax||_\infty=\mu=\max_{1\le i \le n}\sum_{j=1}^{n}|a_{ij}|$
\item $\because ||Ax||_2^2=(Ax, Ax)=x^TA^TAx \ge 0, \forall x \in R^n$ \\
$\therefore A^TA$为非负的对称实矩阵,其特征值均为非负的实数,不妨假设依次排序为: 
$\lambda_1 \ge \lambda_2 \ge \dots \ge \lambda_n \ge 0$\\
对应于一组规范的正交特征向量组$\{u_1, \dots, u_n\}$\\
对于$\forall x \in R^n$,可表示为:$x=\sum^n_{i=1}\alpha_iu_i$\\
对于满足$||x||_2=1$的任意x有:
$$||x||^2_2=(x, x)=\sum^n_{i=1}\alpha^2_i=1$$
$$\begin{array}{lr}
\therefore ||Ax||^2_2=(Ax, Ax)=x^TA^TAx = (A^TAx, x)\\
=(A^TA\sum^n_{i=1}\alpha_iu_i, \sum^n_{i=1}\alpha_iu_i) \\
=(\sum^n_{i=1}\alpha_i \lambda_i u_i, \sum^n_{i=1}\alpha_i u_i)\\
=\sum^n_{i=1}\lambda_i \alpha_i^2  \\
\le \lambda_1 \sum^n_{i=1}\alpha_i^2\\
=\lambda_1\\
\therefore \max_{||x||_2=1}||Ax||_2 \le \sqrt{\lambda_1}
\end{array}$$
特别地,若取$x=u_1$,则:\\
$||Au_1||^2_2=(A^TAu_1, u_1)=\lambda_1$,$\therefore ||Au_1||_2=\sqrt{\lambda_1}$\\
$\max_{||x||_2=1}||Ax||_2 \ge ||Au_1||_2 = \sqrt{\lambda_1}$\\
$\therefore \max_{||x||_2=1}||Ax||_2 = \sqrt{\lambda_1} = \sqrt{\lambda_{\max(A^TA)}}$
\end{itemize}
\end{proof}

推论(自证)若$A\in R^{n\times n}$为对称阵,则$||A||_2=\rho(A)$\\
$A^T=A \Rightarrow \lambda_{A^TA} = \lambda_{A}^2$

\begin{theorem}
\label{theorem_15}
设$||x||_\alpha$为$R^n$上任一种向量范数,从属于它的矩阵范数记为:$||A||_\alpha$,并设$P \in R^{n\times n}, det(P) \neq 0$,则:
\begin{itemize}
\item $R^n$到R的映射$P: x \Rightarrow ||Px||_\alpha$定义了$R^n$上另一种向量范数,记为:$||x||_{P, \alpha}=||Px||_\alpha$
\item 从属向量范数$||x||_{P, \alpha}$的矩阵范数为:\\
$$||A||_{P, \alpha}=||PAP^{-1}_\alpha||$$
\end{itemize}
\end{theorem}

\subsection{矩阵范数的性质}
\begin{theorem}
  \label{theorem:rho}
\begin{itemize}
\item 设$||\cdot||$为$R^{n\times n}$上任一种(从属或非从属)的矩阵范数,则对$\forall A \in R^{n\times n}$,有:
$$\rho(A) \le ||A||$$
\item 对$\forall A \in R^{n\times n}$,及实数$\epsilon > 0$,至少存在一种从属的矩阵范数$||\cdot||$,使得:
$$||A||\le \rho(A)+\epsilon$$
\end{itemize}
\end{theorem}

\begin{proof}
\begin{itemize}
\item 设 $x \neq 0, Ax=\lambda x$,且$|\lambda| = \rho (A)$,则必存在向量$y\in R^n$,使得$xy^T\neq 0$,于是有:
$$\rho(A)||xy^T|| = ||\lambda xy^T|| = ||Axy^T|| \le ||A||\cdot ||xy^T|| \Rightarrow \rho(A) \le ||A||$$
\item 对$\forall A \in R^{n\times n}, \exists T \ s.t. \ TAT^{-1}=J$为Jordan标准形,其中$J=diag(J_1, J_2, \dots, J_s)$,$J_i$为Jordan块:
$$J_i = \left[
\begin{matrix}
\lambda_i &  1 & \ddots\\
~ & \lambda_i & 1  \\
\ddots & \ddots & \lambda_i \\
\end{matrix}
\right], i=1, 2, \dots, s
$$
任取$\varepsilon > 0$,并定义$D_\varepsilon \in R^{n\times n}$为:
$$D_\varepsilon = diag(1, \varepsilon, \dots,\varepsilon^{n-1})$$
易证,$D^{-1}_{\varepsilon}JD_{\varepsilon}$仍为块对角阵,且分块与J相同,即:$ \mathop J\limits^{\sim}=D^{-1}_\varepsilon J D_\varepsilon = diag({\mathop J\limits^{\sim}}_1, {\mathop J\limits^{\sim}}_2, \dots, {\mathop J\limits^{\sim}}_s)$
其中:\\
$${\mathop J\limits^{\sim}}_i = \left[
\begin{matrix}
\lambda_i &  \varepsilon & \ddots\\
~ & \lambda_i & \varepsilon \\
\ddots & \ddots & \lambda_i \\
\end{matrix}
\right], i=1, 2, \dots, s
$$
$$\therefore \  ||{\mathop J\limits^{\sim}}||_\infty = ||D^{-1}_\varepsilon JD_\varepsilon||_\infty \le \max_i |\lambda_i|+\varepsilon = \rho(A)+\varepsilon$$
而$D^{-1}_\varepsilon T=P$为非异阵,由定理\ref{theorem_15}可知:$||D_\varepsilon^{-1}Tx||_\infty$定义了$R^n$上的一种向量范数$||x||_{P, \infty}$且从属于该向量范数的矩阵范数为:
$$||A||=||D^{-1}_\varepsilon TAT^{-1}D_\varepsilon ||_\infty=||D^{-1}_\varepsilon J D_\varepsilon||_\infty \le \rho(A) + \varepsilon$$
\end{itemize}
\end{proof}

\begin{theorem}
对于$R^{n\times n}$的任意两种范数$||\cdot||_\alpha$和$||\cdot||_\beta$,存在常数M和m($M\ge m > 0$)使得:
$$m||A||_\alpha \le ||A||_\beta \le M||A||_\alpha, \forall A \in R^{n\times n}$$
即$R^{n\times n}$上所有矩阵范数是等价的。
\end{theorem}

\begin{theorem}
\label{theorem_43}
设$||\cdot||$是$R^{n\times n}$上的算子范数,若$||B||\le 1$,则$I \pm B$为非异阵,且$||(I+B)^{-1}||\le \frac{1}{1-||B||}$
\end{theorem}

\begin{proof}
 (反证法)\\
若$det(A)=0$则$(I-B)x=0$有非零解,$i.e. \ \exists x_0 \neq 0 {\text{使}} Bx_0 = x_0 \Rightarrow \frac{||Bx_0||}{||x_0||}=1$,而$||B||=\max_{x\neq 0}\frac{||Bx||}{||x||}\ge \frac{||Bx_0||}{||x_o||}=1$,与已知矛盾。\\
$\therefore \ (I-B)$是非异阵,同理可证$I+B$也为非异阵。\\
记$D=(I-B)^{-1}$,则:
$$
\begin{array}{lr}
1 = ||I|| = ||D(I-B)|| = ||D-DB||\ge ||D|| - ||DB|| \ge \\ 
||D|| - ||D||\cdot ||B|| = ||D||\cdot(1-||B||) \Rightarrow ||(I-B)^{-1}|| \le \frac{1}{1-||B||} \\
\end{array}$$
\end{proof}

\section{Gauss 消去法和矩阵的LU分解}
设
\begin{equation}
  Ax=b
\end{equation}
其中:
$$A=\left[
\begin{matrix}
a_{11} & a_{12} & \dots & a_{1n} \\
a_{21} & a_{22} & \dots & a_{2n} \\ 
\dots & \dots & \dots & \dots \\
a_{n1} & a_{n2} & \dots & a_{nn} \\
\end{matrix} \right],
b = \left[ 
\begin{matrix}
b_1 \\ b_2 \\ \dots \\ b_n \\
\end{matrix}
\right],
x = \left[ 
\begin{matrix}
x_1 \\ x_2 \\ \dots \\ x_n \\
\end{matrix}
\right]
$$
且$det(A) \neq 0$,记$[A|b] = [A^{(1)}|b^{(1)}]$其中$A^{(1)} = (a_{ij}^{(1)})$,$b^{(1)}=(b_{i}^{(1)})$

\subsection{Gauss 消去法}
\subsubsection{消元过程}
\begin{itemize}
\item 设$a^{(1)}_{11} \neq 0$,并且选取$a^{(1)}_{11}$作为主元,计算乘数$l_{i1}=a^{(1)}_{i1} / a^{(1)}_{11} (i=2, \dots, n)$;用$-l_{i1}$乘以第1行并加到第i行$(i=2, 3, \dots, n)$,则$[A^{(1)}|b^{(1)}]\Rightarrow [A^{(2)}|b^{(2)}]$
$$[A^{(2)}|b^{(2)}]=\left[
\begin{array}{c c c c c c}
a^{(1)}_{11} & a^{(1)}_{12} & \dots & a^{(1)}_{1n} & \vline & b_1^{(1)} \\
0 & a^{(2)}_{22} & \dots & a^{(2)}_{2n} & \vline & b_2^{(2)} \\
\dots & \dots & \dots & \dots &\vline & \dots \\
0 & a^{(2)}_{n2} & \dots & a^{(2)}_{nn} &\vline & b_n^{(2)} \\ 
\end{array}\right] 
$$
其中,
$$ \left\{
\begin{array}{c c}
a^{(2)}_{ij} = a^{(1)}_{ij} - l_{i1}a^{(1)}_{1j}, &  i, j = 2, 3, \dots, n\\ 
b_i^{(2)} = b_i^{(1)} - l_{i1}b_i^{(1)}, & i=2,3, \dots, n \\
\end{array}\right.
$$
对应的方程$A^{(2)}x=b^{(2)} \Leftrightarrow Ax=b$
\item 假设消元过程已进行了$k-1$步,得到方程组$A^{(k)}x=b^{(k)}$,其对应的增广矩阵为:
\begin{equation}
\label{eqgauss}
[A^{(k)}|b^{(k)}]=\left[
\begin{array}{c c c c c c c c}
a^{(1)}_{11} & a^{(1)}_{12} & \dots & \dots &\dots &a^{(1)}_{1n} & \vline & b_1^{(1)} \\
~ & a^{(2)}_{22} & \ddots & ~ & ~ & ~ &\vline & \vdots \\
~ & ~ & ~ & a^{(k)}_{kk} & \dots & a^{(k)}_{kn} &\vline & b_k^{(k)} \\
~ & ~ & ~ & \vdots        & \dots & \dots         & \vline & \vdots \\
~ & ~ & ~ & a^{(k)}_{nk} & \dots & a^{(k)}_{nn} & \vline & b_n^{(k)} \\ 
\end{array}\right] 
\end{equation}
设$a^{(k)}_{kk}\neq 0$,并选$a^{(k)}_{kk}$为主元素,计算$l_{ik}=\frac{a^{(k)}_{ik}}{a^{(k)}_{kk}}, i=k+1, \dots, n$,分别用$-l_{ik}$乘以第k行并加到第k+1至第n行,则\ref{eqgauss}式可变为:
$$
[A^{(k+1)}|b^{(k+1)}]=\left[
\begin{array}{c c c c c c c c}
a^{(1)}_{11} & a^{(1)}_{12} & \dots & \dots &\dots &a^{(1)}_{1n} & \vline & b_1^{(1)} \\
~ & \ddots & ~ & ~ & ~ & ~ &\vline & \vdots \\
~ & ~ & a^{(k)}_{kk}  & a^{(k)}_{k,k+1}        & \dots & a^{(k)}_{kn} &\vline & b_k^{(k)} \\
~ & ~ & ~                & a^{(k+1)}_{k+1, k+1} & \dots & a^{(k+1)}_{k+1, n} & \vline & b_{k+1}^{(k+1)} \\
~ & ~ & ~                & \vdots                  &  ~   & ~ & \vline & \vdots \\
~ & ~ & ~ & a^{(k+1)}_{n, k+1} & \dots & a^{(k+1)}_{nn} & \vline & b_n^{(k+1)} \\ 
\end{array}\right] 
$$
其中,
$$ \left\{
\begin{array}{c c}
a^{(k+1)}_{ij} = a^{(k)}_{ij} - l_{ik}a^{(k)}_{kj}, &  i, j = k+1, \dots, n\\ 
b_i^{(k+1)} = b_i^{(k)} - l_{ik}b_k^{(k)}, & i=k+1,\dots, n \\
\end{array}\right.
$$
对应的方程组为$A^{(k+1)}x = b^{(k+1)}$
\item 继续这一过程且$a^{(i)}_{ii} \neq 0, i=1, 2, \dots, n-1$(称为主元),直至所有主对角线以下的元素消为0,即可得上三角方程组$A^{(n)}x=b^{(n)}$
$$
[A^{(n)}|b^{(n)}]=\left[
\begin{array}{c c c c c c c}
a^{(1)}_{11} & a^{(1)}_{12} & \dots & \dots & a^{(1)}_{1n}  & \vline & b_1^{(1)} \\
~             & a^{(2)}_{22} & \dots & \dots & a^{(2)}_{2n} &\vline & b_2^{(2)} \\
~             &    ~            & \ddots & ~       & \vdots        &\vline & \vdots \\
~             & ~               & ~ &\ddots     & \vdots        & \vline & \vdots \\
~             & ~               & ~         & ~      & a^{(n)}_{nn} & \vline & b_n^{(n)} \\ 
\end{array}\right] 
$$
\end{itemize}

\subsubsection{回代过程}
$A^{(n)}x=b^{(n)}$的解即为$Ax=b$的解为:
$$ \left\{
\begin{array}{c c}
x_n = b^{(n)}_n / a^{(n)}_{nn} &  ~\\ 
x_k = (b^{(k)}_k - \sum^n_{j=k+1}a^{(k)}_{kj}x_j) / a^{(k)}_{kk}, & k=n-1, n-2, \dots , 1 \\
\end{array}\right.
$$

\begin{theorem}
$a^{(i)}_{ii} \neq 0 (i=1,2,\dots, k)$的充要条件是A的顺序主子式$D_i \neq 0, i=1, \dots, k, k\le n$
\end{theorem}

\begin{proof}[归纳法]:
\begin{itemize}
\item 充分性:当$k=1$时,显然成立,\\
假设定理对$k-1$成立. i.e. 由$D_i \neq 0, (i=1,2, \dots, k-1) \Rightarrow a^{(i)}_{ii} \neq 0, i=1, 2, \dots, k-1$\\
证对k亦成立,i.e. 证$D_i \neq 0, i=1,2,\dots, k \Rightarrow a^{(i)}_{ii} \neq 0, i=1,2,\dots,k$
由归纳假设$a^{(i)}_{ii} \neq 0, i=1,2,\dots, k-1$于是Gauss消去法将$A^{(1)}=A$化为$A^{(k)}$,即
$$
A^{(1)} \rightarrow A^{(k)}=\left[
\begin{array}{c c c c c c}
a^{(1)}_{11} & a^{(1)}_{12} & \dots & \dots &\dots &a^{(1)}_{1n} \\
~ & a^{(2)}_{22} & \ddots & ~ & ~ & ~ \\
~ & ~ & ~ & a^{(k)}_{kk} & \dots & a^{(k)}_{kn} \\
~ & ~ & ~ & \vdots        & \dots & \dots         \\
~ & ~ & ~ & a^{(k)}_{nk} & \dots & a^{(k)}_{nn} \\ 
\end{array}\right]
$$
由线性代数知识,消去过程中的行变换不影响A的顺序主子式的值。
\begin{equation}
\label{eqD}
D_k = \left|
\begin{array}{c c c}
a_{11} & \dots & a_{1k} \\
\dots & \dots & \dots \\
a_{k1} & \dots & a_{kk}
\end{array}\right|
=a_{11}^{(1)}a_{22}^{(2)}\cdots a_{kk}^{(k)} \neq 0
\end{equation}
由假设$D_i \neq 0, i=1,2, \dots, k$, $\therefore $由上式得,
$$a^{(i)}_{ii} \neq 0, i=1, 2, \dots, k$$
\item 必要条件:由\ref{eqD}式可证。
\end{itemize}
\end{proof}

推论:若A的顺序主子式$D_i \neq 0, i=1,2, \dots, n-1$,则$a^{(1)}_{11}=D_1, a^{(i)}_{ii}=D_i/D_{i-1}, i=2,3,\dots, n$(由\ref{eqD}易知)

\begin{theorem}
对于$Ax=b$,其中A非异阵,若A的顺序主子式$D_i \neq 0, i=1,2, \dots, n-1$,则可用Gauss消去法求出方程组的解。
\end{theorem}

\begin{theorem}
A对称正定 $\Rightarrow a^{(k)}_{kk} > 0, k=1,2,\dots, n$
\end{theorem}

\begin{theorem}
A为严格对角占优阵$\Rightarrow a^{(k)}_{kk} \neq 0, k=1,2,\dots, n$
\end{theorem}

\begin{definition}[严格对角占优阵]
$$|a_{kk}|>\sum^n_{j=1, j\neq k}|a_{kj}|, k=1,2,\dots, n$$
\end{definition}

\subsection{矩阵的LU分解}
\begin{definition}[LU分解]
$A=LU$,L为单位下三角矩阵,U为上三角矩阵 
\end{definition}
$$Ax=b \Leftrightarrow LUx=b \Leftrightarrow 
\left\{
\begin{array}{c}
Ly=b \\
Ux=y \\
\end{array}\right.
$$

第k步消元等价于用下列矩阵:
$$
L_k = \left[
\begin{array}{c c c c c}
1 & ~ & ~ & ~ & ~\\
~ & 1 & ~ & ~ & ~\\
~ & -l_{k+1, k} & 1 ~ & ~\\
~ & \vdots & ~  & \ddots & ~\\
~ & -l_{nk} & ~ & ~ & 1 \\
\end{array}\right]
$$
左乘$[A^{(k)}|b^{(k)}]$, i.e. 
$$L_k[A^{(k)}|b^{(k)}]=[A^{(k+1)}|b^{(k+1)}]$$

第$n-1$步消元等价于用$L_{n-1}$左乘$[A^{(n-1)}|b^{(n-1)}]$, i.e. 
$$L_{n-1}[A^{(n-1)}|b^{(n-1)}]=[A^{(n)}|b^{(n)}]$$
因此经过$n-1$次消元后,即矩阵$[A^{(n)}|b^{(n)}]$左乘$L_1, L_2, \dots, L_{n-1}$有:
$$[A^{(n)}|b^{(n)}] = L_{n-1}[A^{(n-1)}|b^{(n-1)}]=\cdots = L_{n-1}L_{n-2}\dots L_{1}[A^{(1)}|b^{(1)}]$$


$$\therefore U \triangleq A^{(n)} = L_{n-1}L_{n-2}\dots L_{1}A^{(1)}, b^{(n)}=L_{n-1}L_{n-2}\dots L_{1}b^{(1)} $$

$$\Rightarrow L^{-1}_{1}L^{-1}_{2}\dots L^{-1}_{n-1}U \triangleq LU \hfill \text{(Doolittle分解)} $$

易证:
$$L^{-1}_k=\left[
\begin{array}{c c c c c c}
1 & ~ & ~ & ~ & ~ & ~ \\
~ & \ddots & ~ & ~ & ~ & ~ \\
~ & l_{k+1, k} & 1 & ~ & ~ & ~ \\
~ & \vdots & ~ & ~ & \ddots & ~ \\
~ & l_{nk} & 1 & ~ & ~ &1 \\
\end{array}
\right]
$$

$$
\therefore L = L^{-1}_{1}L^{-1}_{2}\dots L^{-1}_{n-1} = \left[
\begin{array}{c c c c c c}
1 & ~ & ~ & ~ & ~ & ~ \\
l_{21} & \ddots & ~ & ~ & ~ & ~ \\
l_{31} & l_{3, 2} & 1 & ~ & ~ & ~ \\
\vdots & \vdots & ~ & ~ & \ddots & ~ \\
l_{n1} & l_{n2} & 1 & ~ & ~ &1 \\
\end{array}
\right], |A| \neq 0
$$

\begin{theorem}[LU分解定理]
\label{th_lu}
非奇异阵$A\in R^{n\times n}$,若其顺序主子式$D_i \neq 0 (i = 1, 2, \dots, n-1)$,则存在唯一的单位下三角阵L和上三角阵U,使得$A=LU$
\end{theorem}
\begin{proof}[反证法]
假设A的LU分解不唯一,i.e. $\exists A=L_1U_1 = L_2U_2 \Rightarrow U_1 = L_1^{-1}L_2U_2\Rightarrow U_1 U_2^{-1}=L_1^{-1}L_2 = I \Rightarrow L_1 = L_2, U_1 = U_2$
\end{proof}
注释:
\begin{itemize}
\item 当A为奇异阵,且$D_i \neq 0 (i = 1, 2, \dots, n-1)$时,定理\ref{th_lu}仍成立。
\item A的k阶($k=1, \dots, n$)顺序主子式
$$D_k = \left|
\begin{matrix}
a_{11} & \dots & a_{1k} \\
\dots & \dots & \dots \\
a_{k1} & \dots & a_{kk} \\
\end{matrix}
\right| = a^{(1)}_{11}a^{(2)}_{22}\cdots a^{(k)}_{kk} = u_{11}u_{22}\cdots u_{kk}
$$
\item 若A的顺序主子式$D_k \neq 0 (k=1, 2, \dots, n)$,则存在唯一的下三角阵${\mathop L\limits^{\sim}}$和单位上三角阵${\mathop U\limits^{\sim}}$使得:
\begin{equation*}
A={\mathop L\limits^{\sim}}{\mathop U\limits^{\sim}}
\end{equation*} 
称为Crout分解。
\item \begin{theorem}[LDU分解定理]
设$A \in R^{n\times n}$,则存在唯一的单位下、上三角阵L、U及对角阵D使得$A=LDU$的充要条件是A的顺序主子式$D_k \neq 0, k=1,2,\dots, n-1$,$A=L{\mathop U\limits^{\sim}}$,而${\mathop U\limits^{\sim}} = DU$
\end{theorem}
\end{itemize}

\section{主元素消去法}
\subsection{列主元Gauss消去法}
Gauss消去过程:
$$
A^{(k)}=\left[
\begin{matrix}
a^{(1)}_{11} & a^{(1)}_{12} & \dots & \dots &\dots &a^{(1)}_{1n} \\
~ & a^{(2)}_{22} & & & & \vdots\\
& & \ddots & & & \vdots \\
~ & ~ & ~ & a^{(k)}_{kk} & \dots & a^{(k)}_{kn} \\
~ & ~ & ~ & \vdots        & \dots & \dots         \\
~ & ~ & ~ & a^{(k)}_{nk} & \dots & a^{(k)}_{nn} \\ 
\end{matrix}\right]
$$
需要列主元的两种情况:
\begin{itemize}
\item $a^{(kk)}_{kk} = 0$
\item $|a^{(kk)}_{kk}|$很小,$l_{ik}=\frac{a^{(k)}_{ik}}{a^{(k)}_{kk}} > 1$会放大误差
\end{itemize}
\subsubsection{消去过程}
对$k=1, 2, \dots, n-1$ 做:
\begin{itemize}
\item 选列主元:$$|a_{i_kk}^{(k)}| = \max_{k \le i \le n} |a^{(k)}_{ik}|$$
\item 若$a_{i_kk}^{(k)} = 0$,则停止计算($\because det(A) = 0$), 否则:
\item 若$i_k \neq k$,则换行:$a_{kj}^{(k)} \leftrightarrow a_{i_kk}^{(k)}$,$b^{(k)}_k \leftrightarrow b^{(k)}_{i_k}$
\item 消元:对$i=k+1, \dots, n$做(a) - (c)
$$
(a) l_{ik} = a_{ik}^{(k)} / a_{kk}^{(k)} 
$$
对$j=k+1, \dots, n$做(b)
$$(b) a_{ij}^{(k+1)} = a_{ij}^{(k)} - l_{ik}a_{kj}^{(k)}, b^{(k+1)}_i = b^{(k)}_i - l_{ik}b^{(k)}_k $$
\end{itemize}

\subsubsection{回代过程}
\begin{itemize}
\item 若$a^{(n)}_{nn} = 0$,则停止计算($\because det(A) = 0$),否则
\item $x_n = b^{(n)}_n / a^{(n)}_{nn}$
\item 对$i=n-1, n-2, \dots, 1$做:
$$x_i = (b^{(i)}_i - \sum^n_{j=j+1}a^{(i)}_{ij}x_j) / a^{(i)}_{ii}$$
\end{itemize}

\subsection{完全选主元消去法}
$$|a^{(k)}_{i_kj_k}| = \max_{\substack{k \le i \le n \\ k\le j \le n}} |a^{(k)}_{ij}|$$

\section{直接分解法}

\begin{theorem}
  若A为非奇异阵,则$\exists$排列阵P(初等置换阵乘积),使得$PA=LU$其中L为单位下三角阵,U为上三角阵$A=LU$
\end{theorem}

\begin{equation*}
  Ax=b \Leftrightarrow LUx = b \Leftrightarrow \left\{
    \begin{array}{lr}
      Ly = b \\
      Ux = y 
    \end{array}
    \right.
\end{equation*}

\subsection{Doolittle分解法}
假设A非异阵,且$A=LU$,i.e.
$$
A = \left[
  \begin{matrix}
    a_{11} & a_{12} & \dots & a_{1n} \\
    a_{21} & a_{22} & \dots & a_{2n} \\
    \dots \\
    a_{n1} & a_{n2} & \dots & a_{nn}
  \end{matrix}\right]=\left[
    \begin{matrix}
      1 \\
      l_{21} & 1 \\
      \dots & \dots & \ddots \\
      l_{n1} & l_{n2} & \dots & 1 
    \end{matrix}\right]
    \left[
      \begin{matrix}
        u_{11} & u_{12} & \dots & u_{1n} \\
        ~ & u_{22} & \dots & u_{2n} \\
        ~ & ~ & \ddots & \vdots \\
        ~ & ~ & ~ & u_{nn}
      \end{matrix}
      \right]
$$
对于L和U,由于$i<j$时,$l_{ij}=0$,当$i>j, u_{ij}=0$,$\therefore$由$A=LU$可得:
\begin{equation}
  a_{kj} = \sum^n_r=1l_{kr}u_{rj} = \sum_{r=1}^{\min(k,j)}l_{kr}u_{kj}, k,j = 1,2,\dots, n
  \label{eq:4.1}
\end{equation}
显然:
\begin{equation}
  u_{1j}=a_{1j},j=1,2,\dots, n 
  \label{eq:4.2}
\end{equation}

当$j=1$时,由\ref{eq:4.1}可得,
\begin{equation}
  a_{k1}=l_{k1}u_{11}\Rightarrow l_{k1}=a_{k1}/u_{11}, k=2,3,\dots n 
  \label{eq:4.3}
\end{equation}

假设U的第1行至第$k-1$行和L的第1列至第$k-1$列都已求出,则可计算U的第k行元素
\begin{equation}
  \begin{array}{lr}
    \because a_{kj}=l_{kk}u_{kj} + \sum^{k-1}_{r=1}l_{kr}u_{rj} \\
    \therefore u_{kj}=a_{kj} - \sum^{k-1}_{r=1}l_{kr}u_{rj}, j=k,k+1,\dots, n
  \end{array}
  \label{eq:4.4}
\end{equation}

\begin{equation}
  \begin{array}{lr}
    \because a_{ik} = l_{ik}u_{kk} + \sum^{k-1}_{r=1}l_{ir}u_{rk} \\
    \therefore l_{ik} = (a_{ik}-\sum^{k-1}_{r=1}l_{ik}u_{rk})/u_{kk}, i=k+1, \dots, n
  \end{array}
  \label{eq:4.5}
\end{equation}

利用(\ref{eq:4.2})(\ref{eq:4.3})(\ref{eq:4.4})(\ref{eq:4.5})将A分解为LU,这种分解方法称为Doolittle三角分解。

\begin{equation}
  Ax=b \Leftrightarrow LUx = b \Leftrightarrow \left\{
    \begin{array}{lr}
      Ly = b \\
      Ux = y 
    \end{array}
    \right. \Rightarrow y_i=b_i-\sum^{i-1}_{r=1}l_{ir}y_r, i=1,2,\dots, n
    \label{eq:4.6}
\end{equation}
$i=1$时,$\sum^0_{r=1}l_{ir}y_r=0$

\begin{equation}
  x_i = (y_i-\sum^n_{r=i+1}u_{ir}x_r)/u_{ii}, i=n,n-1,\dots,1
  \label{eq:4.7}
\end{equation}

首先用(\ref{eq:4.2})(\ref{eq:4.3})(\ref{eq:4.4})(\ref{eq:4.5})将A分解为LU,再用(\ref{eq:4.6})(\ref{eq:4.7})求出方程组的解,称为Doolitle三角分析方法。

\subsection{三对角方程组的追赶法}
设有方程$Ax=d$,且$D_i\neq 0,i=1,2,\dots, n$其中:
$$A=\left[
  \begin{matrix}
    b_1 & c_1 \\
    a_2 & b_2 & c_2 \\
    ~ & \ddots & \ddots & \ddots \\
    ~ & ~ & \ddots & \ddots & c_{n-1} \\
    ~ & ~ & & a_n & b_n
  \end{matrix}
  \right]$$
设$A=LU$,易证:
$$L=\left[
  \begin{matrix}
    1 \\
    l_2 & 1 \\
    & \ddots & \ddots \\
    & & \ddots & \ddots \\
    & & & l_n & 1
  \end{matrix}
  \right], U = \left[
    \begin{matrix}
      u_1 & c_1 \\
      & u_2 & c_2 \\
      & & \ddots & \ddots \\
      & & & \ddots & c_{n-1} \\
      & & & & u_n
    \end{matrix}
    \right]
$$
$\therefore$ 由$A=LU$即可确定:
\begin{equation}
  \left\{
  \begin{array}{lr}
    u_1 = b_1 \\
    l_i = a_i/u_{i-1}, i=2,3,\dots, n \\
    u_i = b_i-l_ic_{i-1}, i=2,3,\dots,n
  \end{array}\right.
  \label{eq:4.8}
\end{equation}

\begin{equation}
  LUx = b \Leftrightarrow \left\{
    \begin{array}{lr}
      Ly = b \\
      Ux = y 
    \end{array}
    \right. \Rightarrow \left\{
    \begin{array}{lr}
      y_1 = d_1 \\
      y_i=d_i-l_iy_{i-1}, i=2,3,\dots, n
    \end{array}\right.
    \label{eq:4.9}
\end{equation}

\begin{equation}
  \left\{
    \begin{array}{lr}
      x_n = y_n/u_n \\
      x_i = (y_i-c_ix_{i+1})/u_i, i=n-1,n-2,\dots, 1
    \end{array}
    \right.
    \label{eq:4.10}
\end{equation}

\begin{theorem}
  设三对角阵A满足对角占优条件,$|b_1|>|c_1|>0, |b_n|>|a_n|>0, |b_i|\ge |a_i|+|c_i|$则A非奇异,且$y_i\neq 0, i=1,2,\dots, n$
  $0<|\frac{c_i}{u_i}|<1, i=1,2,\dots, n-1, |b_i|-|a_i|<|u_i|<|b_i|+|a_i|, i=2,3,\dots, n$
\end{theorem}

\subsection{Cholesky分解(平方根)法}
\textbf{对称}:$A^T = A$;\textbf{正定}:$\lambda_i > 0$,$D_i > 0$

假设$det(A)\neq 0$,且$D_i\neq 0, i=1,2,\dots, n-1$
则$A=LU$进一步:
$$U=\left[
  \begin{matrix}
    u_{11} \\
    & u_{22} \\
    & & \ddots \\
    & & & u_{nn}
  \end{matrix}\right]
  \left[
    \begin{matrix}
      1 & u_{12}/u_{11} & \dots & u_{1n}/u_{11} \\
      & 1 & \dots & u_{2n}/u_{22} \\
      & & \ddots & \vdots \\
      & & & 1
    \end{matrix}
    \right]=D\widetilde{U}
  $$
i.e. $A=LD\widetilde{U}$这就是A的LDU分解。

\begin{theorem}
  (对称阵的三角分解定理)设$A\in R^{n\times n}$,且A的顺序主子式$D_i\neq 0, i=1,2,\dots,n$则$\exists$唯一单位下三角阵L和对角阵D使得$A=LDL^T$
\end{theorem}

\begin{proof}
$\because A=A^T=LD\widetilde{U} = \widetilde{U}^TDL^T \stackrel{\text{分解唯一性}}{\Rightarrow} L=\widetilde{U}^T\Rightarrow \widetilde{U}=L^T$ \\
$\therefore A=LD\widetilde{U}=LDL^T$
\end{proof}

\begin{theorem}
  设$A\in R^{n\times n}$,且A对称正定,则$\exists$唯一的对角元素为正的下三角阵L,使得$A=LL^T$,由
  $$A=LL^T, L = \left[
    \begin{matrix}
      l_{11} \\
      l_{21} & l_{22} \\
      \vdots & \vdots & l_{33} \\
      \vdots & \vdots & &\ddots \\
      l_{n1} & l_{n2} & \dots & \dots & l_{nn}
    \end{matrix}
    \right]$$
  $$a_{ij}=(\sum_{k=1}^jl_{ik}l_{jk})=\sum^{j-1}_{k=1}l_{ik}l_{jk}+l_{ij}l_{jj}, i=j,j+1,\dots,n $$
  $\Rightarrow$
  \begin{align}
    &l_{jj} = \sqrt{a_{jj}-\sum^{j-1}_{k=1}l^2_{jk}}, j=1,2,\dots, n \\
    &l_{ij} = (a_{ij}-\sum^{j-1}_{k=1}l_{ik}l_{jk})/l_{jj}, i=j+1, \dots, n, j=1,2, \dots, n-1 
  \end{align}
  方程组的解:
  \begin{align}
    &y_i = (b_i-\sum^{i-1}_{k=1}l_{ik}y_k)/l_{ii}, i=1,2,\dots, n \\
    &x_i = (y_i-\sum^n_{k=i+1}l_{ki}x_k)/l_{ii}, i=n,n-1,\dots, 1
  \end{align}
\end{theorem}

\begin{proof}
  A对称正定$\therefore D_i=u_{11}u_{22}\dots u_{ii}>0 \therefore u_{ii}>0, i=1,2,\dots,n $ \\
  $\therefore D = D^\frac{1}{2}D^\frac{1}{2}$,其中$D^\frac{1}{2} = diag(\sqrt{u_{11}}, \sqrt{u_{22}}, \dots, \sqrt{u_{nn}}) $ \\
  $\therefore A= LD^\frac{1}{2}D^\frac{1}{2}L^T = (LD^\frac{1}{2})(LD^\frac{1}{2})^T = \widetilde{L}\widetilde{L}^T $\\
  由$A=LDL^T$的分解唯一性$\Rightarrow A=\widetilde{L}\widetilde{L}^T$也是唯一的。
\end{proof}

\section{条件数和摄动理论初步}
\begin{example}
  $$\left\{
    \begin{array}{lr}
      x_1 + x_2=2\\
      x_1+1.00001x_2=2
    \end{array}
    \right.
    \Rightarrow x_1=2, x_2=0$$
    若上式变为:
    $$
    \left\{
    \begin{array}{lr}
      x_1 + x_2=2\\
      x_1+1.00001x_2=2.00001
    \end{array}
    \right.
    \Rightarrow x_1=1=x_2
    $$
    这就是病态方程组。
\end{example}

\begin{definition}
  设$Ax=b$,当A和b有微小变化$\Delta A$和$\Delta b$就引起解向量x的很大变化,
  称A为关于解方程组和矩阵求逆的病态矩阵。称相应的方程组为病态方程组,
  反之,若$\Delta A$和$\Delta b$很小,$\Delta x$也很小,就称A为良态矩阵和称$Ax=b$为良态方程组。
\end{definition}

\subsection{右端项b的摄动对解的影响}
假设$det(A)\neq 0$,b有扰动$\Delta b$,则$Ax=b$的解也要产生摄动$\Delta x$,i.e.方程组$Ax=b$变成了$A(x+\Delta x)=b+\Delta b\Rightarrow 
\Delta x =A^{-1}\Delta b \Rightarrow ||\Delta x||\le ||A^{-1}||\cdot||\Delta b|| $ \\
$\because ||x||\ge \frac{||Ax||}{||A||}=\frac{||b||}{||A||}\Rightarrow \frac{1}{||x||}\le \frac{||A||}{||b||}$ \\
$\therefore \frac{||\Delta x||}{||x||} \le ||A||\cdot||A^{-1}||\frac{||\Delta b||}{||b||}$

\subsection{A的摄动对解的影响}
设$det(A)\neq 0$,A有摄动$\Delta A$,相应地由$x\rightarrow x+\Delta x$,i.e. 
$Ax=b, (A+\Delta A)(x+\Delta x)=b\Rightarrow (A+\Delta A)\Delta x = -\Delta Ax\Rightarrow A(I+A^{-1}\Delta A)\Delta x = -\Delta Ax$

当$||A^{-1}||\cdot||\Delta A||<1$时,由矩阵范数定理\ref{theorem_43}($||(I\pm B)^{-1}||\le \frac{1}{1-||B||} (||B||<1)$),可知,$(I+A^{-1}\Delta A)^{-1}$存在
$$\therefore \Delta x = -(I+A^{-1}\Delta A)^{-1}A^{-1}\Delta Ax $$
$$\therefore ||\Delta x||\le ||(I+A^{-1}\Delta A)^{-1}||||A^{-1}||||\Delta A||||x||\le \frac{||A^{-1}||||\Delta A||}{1-||A^{-1}\Delta A||}||x|| $$
$$\Rightarrow \frac{||\Delta x||}{||x||} \le \frac{||A||||A^{-1}||\cdot\frac{||\Delta A||}{||A||}}{1-||A||||A^{-1}||\frac{||\Delta A||}{||A||}} = \frac{cond(A)\frac{||\Delta A||}{||A||}}{1-cond(A)\frac{||\Delta A||}{||A||}}$$

\begin{definition}
  对于非异阵A,$||\cdot||$为一种矩阵的从属范数,称数$||A||\cdot||A^{-1}||$为A的条件数,记为$cond(A)=||A||\cdot||A^{-1}||$
\end{definition}

常用条件数:
$$cond(A)_\infty =||A||_\infty\cdot||A^{-1}||_\infty $$
$$cond(A)_2 = ||A||_2||A^{-1}||_2=\sqrt{\frac{\lambda_{\max}(A^TA)}{\lambda_{\min}(A^TA)}}$$
特别地当:$A^T=A$时,$cond(A)_2=\frac{|\lambda_1|}{|\lambda_n|}$,其中$\lambda_1$和$\lambda_n$分别为A的绝对值最大和最小的特征值。

条件数性质:(运用线性代数知识证明)
\begin{enumerate}
  \item $cond(A)\ge 1, cond(A)=cond(A^{-1})$
  \item $cond(cA) = cond(A), \forall c\in R, c\neq 0$
  \item 若矩阵A正交阵,则$cond(A)_2=1$
  \item 若U为正交阵,则$cond(A)_2=cond(AU)_2=cond(UA)_2$
  \item $cond(A)\ge\frac{|\lambda_1|}{|\lambda_n|}$,其中$\lambda_1, \lambda_n$分别为绝对值最大,最小的特征值。(证明:定理\ref{theorem:rho})
\end{enumerate}

\begin{theorem}
  设$Ax=b,det(A)\neq 0$,b为非零向量且$\Delta b$和$\Delta A$分别为b和A的扰动量,若$||A^{-1}||\cdot||\Delta A||<1$则有事前误差估计式:
  \begin{equation}
    \frac{||\Delta x||}{||x||} \le \frac{cond(A)}{1-cond(A)\frac{||\Delta A||}{||A||}}[\frac{||\Delta b||}{||b||}+\frac{||\Delta A||}{||A||}]
  \end{equation}
  \label{theorem:5.1}
\end{theorem}

\begin{example}
  已知方程组$\left\{\begin{array}{lr}
    x_1+x_2=2\\
    x_1+1.000001x_2=2
  \end{array}\right.$,右端项b有扰动$\Delta b=(0, 10^{-5})^T$,求其系数矩阵A的条件数$cond(A)_\infty$,说明$\Delta b$对解的影响,并分析其性态。

  解:$\because A=\left[\begin{matrix}
    1 & 1 \\
    1 & 1+10^{-6}
  \end{matrix}\right], A^{-1}=\left[\begin{matrix}
    1+10^6 & -10^6 \\
    -10^6 & 10^6
  \end{matrix}\right] $\\
  $\therefore cond(A)_\infty =||A||_\infty||A^{-1}||_\infty = (2+10^{-6})(1+2\times 10^6)>4\times 10^6$
  由于条件数很大,方程组病态,A为病态矩阵。

  右端项对解的影响:$\frac{||\Delta x||}{||x||} \le cond(A)\frac{||\Delta b||}{||b||}= 2000\%$
\end{example}

\begin{theorem}
  设$det(A)\neq 0$,x和$\bar{x}$分别为$Ax=b$的准确解和近似解,$r=b-A\bar{x}$为残差。则:
  \begin{equation}
    \frac{1}{cond(A)}\frac{||r||}{||b||}\le \frac{||x-\bar{x}||}{||x||}\le cond(A)\frac{||r||}{||b||}
  \end{equation}
  称为事后误差估计式。
  \label{theorem:after}
\end{theorem}

\begin{proof}
  $\because x=A^{-1}b, \bar{x}=A^{-1}(b-r) \therefore x-\bar{x}=A^{-1}r $\\
  \begin{equation*}
    \begin{split}
      \frac{1}{cond(A)}\frac{||r||}{||b||} &= \frac{||AA^{-1}r||}{||A||\cdot||A^{-1}||\cdot||b||} \\
      &\le \frac{||A||||A^{-1}r||}{||A||\cdot||A^{-1}||\cdot||b||} = \frac{||A^{-1}r||}{||A^{-1}||\cdot||b||} \\
      &\le \frac{||x-\bar{x}||}{||x||} \\
      &\le \frac{||A^{-1}||||r||||A||}{||A^{-1}b||||A||} \le cond(A)\frac{||r||}{||AA^{-1}b||} \\
      &= cond(A)\frac{||r||}{||b||}
    \end{split}
  \end{equation*}
\end{proof}

\section{迭代法的基本概念}
\subsection{迭代法的一般形式}
对于方程:
\begin{equation}
  Ax=b, det(A)\neq 0, b\in R^n
  \label{eq:6.1}
\end{equation}
将\ref{eq:6.1}转化为等价的方程组:
\begin{equation}
  x= Bx+f, B\in R^{n\times n}, f\in R^n
  \label{eq:6.2}
\end{equation}
由\ref{eq:6.2}可构造如下迭代公式:
\begin{equation}
  x^{(k+1)}=Bx^{(k)}+f, k=0,1,2,\dots
  \label{eq:6.3}
\end{equation}
其中:B称为迭代矩阵,f为右端项,给定$x^{(0)}$则可利用\ref{eq:6.3}得向量序列:$\{x^{(k)}\},x^{(k)}=(x_1^{(k)}, x_2^{(k)}, \dots, x_n^{(k)}) $,
若$x^{(k)}$收敛,i.e.$x^{(k)}\rightarrow x^*, k\rightarrow \infty$,则有:
$$x^*=Bx^*+f $$

\subsection{向量序列和矩阵序列的收敛性}
\begin{definition}
  设$\{x^{(k)}\}$为$R^n$中的向量序列,$x^*\in R^n$,若$\lim_{k\rightarrow \infty}||x^{(k)} - x^*||=0 $,
  其中$||\cdot||$为向量范数,则称$\{x^{(k)}\}$收敛于$x^*$,记为$\lim_{k\rightarrow \infty}x^{(k)} = x^* $。
\end{definition}

\begin{theorem}
  $R^n$中的$\{x^{(k)}\}$收敛于$x^*\in R^n$,当且仅当$\lim_{k\rightarrow \infty}x_i^{(k)}=x_i^*, i=1,2,\dots, n $。
  其中$x^{(k)}=(x_1^{(k)}, x_2^{(k)}, \dots, x_n^{(k)})^T, x^*=(x_1^*, x_2^*, \dots, x_n^*)^T $
\end{theorem}

\begin{proof}
  若$\{x^{(k)}\}$收敛,则$\lim_{k\rightarrow \infty}||x^{(k)}-x^*|| = 0 $,又因为:$0\le |x_i^{(k)}-x_k^*|\le \max_{n\ge i\ge 1}|x_i^{(k)}-x_k^*|=||x^{(k)}-x^*||_\infty \rightarrow 0 $\\
  $\therefore \lim_{k\rightarrow \infty}x^{(k)} = x^*$

  反之,若$\lim_{k\rightarrow \infty}x_i^{(k)}=x_i^*, i=1,2,\dots, n $,则可证:$\lim_{k\rightarrow \infty}||x^{(k)}-x^*||_\infty = 0$
  又因为向量范数的等价性\ref{def:equal},有$||x^{(k)}-x^*||\le C||x^{(k)}-x^*||_\infty \rightarrow 0$
\end{proof}

\begin{definition}
  定义了范数$||\cdot||$的空间$R^{n\times n} $中,若$\exists A \in R^{n\times n}$,使$\lim_{k\rightarrow \infty}||A^{(k)-A}||=0$,则称$\{A^{(k)}\}$收敛于
  A,记为:$\lim_{k\rightarrow \infty} A^{(k)}=A$
\end{definition}

\begin{theorem}
  【自证】设$A^{(k)}=(a_{ij}^{(k)})_n, (k=1,2,\dots, \infty), A=(a_{ij})_n$为$R^{n\times n}$中的矩阵,则
  $\{A^{(k)}\}$收敛于A的充要条件为:
  $$\lim_{k\rightarrow \infty}a_{ij}^{(k)}=a_{ij}$$
\end{theorem}

\begin{theorem}
  $$\lim_{k\rightarrow \infty} A^{(k)}=0 \Leftrightarrow \lim_{k\rightarrow \infty} A^{(k)}x=0, \forall x \in R^n$$
  \label{theorem:6.3}
\end{theorem}
\begin{proof}
  \begin{itemize}
    \item 必要性:对任一从属范数:$||A^{(k)}x||\le ||A^{(k)}||||x|| \rightarrow 0$
    \item 充分性:取$x=e_j$,则由$\lim_{k\rightarrow \infty}A^{(k)}e_j = 0$,可知$A^{(k)}$的第j列为0,当$j=1,2,\dots, n$时,即可证明$\lim_{k\rightarrow \infty}A^{(k)}=0$
  \end{itemize}
\end{proof}

\begin{theorem}
  设$B\in R^{n\times n} $则有下列三个命题等价: 
  \begin{itemize}
    \item $\lim_{k\rightarrow \infty} B^k=0$
    \item $\rho(B)<1$
    \item 至少存在一种从属矩阵范数$||\cdot||$,使得$||B||<1$
  \end{itemize}
\end{theorem}

\begin{proof}
  逐个证明:

  (1)->(2)反证法:假设B有一个特征值$\lambda$满足$|\lambda|\ge 1$,则有$\lambda \neq 0$使得$Bx=\lambda x\Rightarrow B^kx=\lambda B^{k-1}x=\lambda^kx,\therefore k->\infty,B^kx\neq 0 $
  根据$(\lim_{k\rightarrow \infty} A^{(k)}=0\Leftrightarrow \lim_{k\rightarrow \infty} A^{(k)}x=0, \forall x \in R^n )$得,$\lim_{k\rightarrow \infty} B^{k}\neq 0$与已知矛盾,$\therefore \rho(B)<1$

  (2)->(3)由定理\ref{theorem:rho}知,对$\forall \varepsilon >0, \exists ||\cdot|| s.t. ||B||\le \rho(B)+\varepsilon \Rightarrow ||B||<\rho(B)+2\varepsilon, \because \rho(B)<1, \text{choose\ } \varepsilon = \frac{1-\rho(B)}{2}>0
  \therefore ||B||<\rho(B)+2\varepsilon = \rho(B)+2\frac{1-\rho(B)}{2}=1 $

  (3)->(1)由相容性条件可得:$||B^k-0||\le ||B||^k \because ||B||<1 \Rightarrow \lim_{k\rightarrow \infty}||B||^k=0 \Rightarrow \lim_{k\rightarrow \infty}B^k=0 $
\end{proof}

\subsection{迭代方法的收敛性}
$x^{(k+1)}=Bx^{(k)}+f $,若收敛,则有$x^*=Bx^*+f$,则$e^{(k)}=x^{(k)}-x^*=Bx^{(k-1)}-Bx^*= \dots = B^ke^{(0)}, e^{(0)}=x^{(0)}-x^* $

\begin{theorem}
  \textbf{(重要)}对于任意初值$x^{(0)} $和右端项f,迭代方法$x^{(k+1)}=Bx^{(k)}+f $收敛的\textbf{充要条件}是:$\rho(B)<1$
\end{theorem}
\begin{proof}
  由以上分析有:$e^{(k)}=x^{(k)}-x^*= B^ke^{(0)}$,其中$e^{(0)}$为与k无关的任意初始误差\\
  $\therefore$迭代法的收敛性等价于$\{e^{(k)} \}$的收敛性。\\
  而由$e^{(0)}$的任意性,及定理\ref{theorem:6.3}知:$\{e^{(k)} \}$的收敛性等价于$\{B^k\}$的收敛性,而$\{B^k\}$的收敛性与
  $\rho(B)<1$等价\\
  $\therefore$迭代法收敛$\Leftrightarrow\rho(B)<1$
  $$x^{(k)}\rightarrow x^* \Leftrightarrow B^ke^{(0)}\rightarrow 0\Leftrightarrow B^k\rightarrow 0 \Leftrightarrow \rho(B)<1 $$
\end{proof}

\begin{theorem}
  \textbf{(充分条件)}对于$x^{(k+1)}=Bx^{(k)}+f $,若$\exists ||\cdot||_\lambda, s.t.\ ||B||_\lambda < 1$则:
  \begin{enumerate}
    \item 迭代法收敛;i.e. $x^{(k)}\rightarrow x^*, k\rightarrow \infty$
    \item $$||x^{(k)}-x^*||_\lambda \le \frac{||B||_\lambda}{1-||B||_\lambda}||x^{(k)}-x^{(k-1)}||_\lambda$$
    \item $$||x^{(k)}-x^*||_\lambda \le \frac{||B||^k_\lambda}{1-||B||_\lambda}||x^{(0)}-x^*||_\lambda$$
  \end{enumerate}
\end{theorem}
\begin{proof}
  \label{theorem:iter}
  \begin{enumerate}
    \item $\rho(B) < ||B|| < 1$
    \item $\because x^{(k)}-x^* = B(x^{(k-1)}-x^*)=B(x^{(k)}-x^*)+B(x^{(k-1)}-x^{(k)})
    \Rightarrow (I-B)(x^{(k)}-x^*)=B(x^{(k-1)}-x^{(k)})$\\
    $\because||B||<1, \therefore \exists (I-B)^{-1} 
    \therefore x^{(k)}-x^*=-(I-B)^{-1}B(x^{(k)}-x^{(k-1)}) $\\
    $\therefore ||x^{(k)}-x^*||_\lambda \le ||(I-B)^{-1}
    ||_\lambda||B||_\lambda||x^{(k)}-x^{(k-1)}||_\lambda\le \frac{||B||_\lambda}{1-||B||_\lambda}||x^{(k)}-x^{(k-1)}||_\lambda$
    \item $\because ||x^{(k)}-x^{(k-1)}||_\lambda =||B(x^{(k-1)}-x^{(k-2)})||_\lambda \le ||B||_\lambda|||x^{(k-1)}-x^{(k-2)}||_\lambda \le \dots \le ||B||^{k-1}_\lambda||x^{(1)}-x^{(0)}||_\lambda $\\
      将上式代入2中,即可得到3.
  \end{enumerate}
\end{proof}

注释:
\begin{itemize}
  \item 若仅仅是为了判断迭代算法的收敛性,则定理中的条件还可放宽为:$\exists$某一种范数使得$||B||_\lambda<1$
  \item 方法的收敛性与右端项f无关
  \item 从定理可看出,$||B||_\lambda$不是很靠近1,如果要求$||x^{(k)}-x^*||_\lambda<\varepsilon$,只需要使相邻两次的$||x^{(k)}-x^{(k-1)}||_\lambda < \varepsilon$即可
  \item 判断一种方法的迭代次数,用定理\ref{theorem:iter}.3可以解析求出。
\end{itemize}

\begin{definition}
  \textbf{(重要概念)}称$R(B)=-ln(\rho(B)) $为迭代法$x^{(k+1)}=Bx^{(k)}+f $的渐进收敛率或渐进收敛速度。
\end{definition}

\section{Jacobi方法和GS迭代法}
\subsection{J法}
假设$det(A)\neq 0,A=D-L-U$,其中:$D=diag(a_{11}, a_{22}, \dots, a_{nn})$
$$L=\left[
  \begin{matrix}
    0 \\
    -a_{21} & 0 \\
    \vdots & \vdots & \ddots \\
    -a_{n1} & -a_{n2} & \dots & 0
  \end{matrix}
  \right],
  U=\left[\begin{matrix}
    0 & -a_{12} & \dots & -a_{1n} \\
      &  \ddots & \dots & -a_{2n} \\
      & & \ddots & \vdots \\
      & & & 0
  \end{matrix}
    \right]
  $$

将$A=D-L-U$代入$Ax=b$则$Dx=(L+U)x+b\Rightarrow x=D^{-1}[(L+U)x+b]=D^{-1}(L+U)x+D^{-1}b \Rightarrow B=D^{-1}(L+U), f=D^{-1}b $

由此可得J法:
\begin{equation}
  \begin{array}{lr}
    x^{(k+1)} = Bx+f \\
    B = D^{-1}(L+U) = I-D^{-1}A \\
    f = D^{-1}b
  \end{array}
\end{equation}

其分量形式:
\begin{equation}
  x_i^{(k+1)} = \frac{1}{a_{ii}}[b_i-\sum_{j=1}^{i-1}a_{ij}x_j^{(k)}-\sum^n_{j=i+1}a_{ij}x^{(k)}_j],j=1,2,\dots, n
\end{equation}

\subsection{GS方法}
分量形式:
\begin{equation}
  x_i^{(k+1)} = \frac{1}{a_{ii}}[b_i-\sum_{j=1}^{i-1}a_{ij}\mathbf{x_j^{(k+1)}}-\sum^n_{j=i+1}a_{ij}x^{(k)}_j],j=1,2,\dots, n
\end{equation}

写成向量形式有:
\begin{equation}
  \begin{array}{lr}
    x^{(k+1)} = D^{-1}(b+Lx^{(k+1)}+Ux^{(k)}) \\
    \Leftrightarrow (D-L)x^{(k+1)} = b + Ux^{(k)} \\
    \Leftrightarrow x^{(k+1)} = (D-L)^{-1}Ux^{(k)} + (D-L)^{-1}b \\
    = (I-(D-L)^{-1}A)x^{(k)} + (D-L)^{-1}b
  \end{array}
\end{equation}

\begin{example}
  $$\left\{
    \begin{array}{lr}
      10x_1-x_2-2x_3=72 \\
      -x_1+10x_2-2x_3=83 \\
      -x_1-x_2+5x_3=42
    \end{array}
    \right.$$
    易知解$x^*=(11,12,13)$

    解:J法:
    $$\begin{array}{lr}
      x_1^{(k+1)} = \frac{1}{10}(72 + x_2^{(k)}+2x_3^{k}) \\
      x_2^{(k+1)} = \frac{1}{10}(83 + x_1^{(k)}+2x_3^{(k)}) \\
      x_3^{(k+1)} = \frac{1}{5}(42+x_1^{(k)}+x_2^{(k)})
    \end{array}$$
    取$x^{(0)}=(0,0,0)^T $则:$x^{(9)}=(10.9994, 11.9994, 12.9992)^T $,误差$||x^{(9)}-x^*||_\infty = 0.0008 $

    GS法:
    $$\begin{array}{lr}
      x_1^{(k+1)} = \frac{1}{10}(72 + x_2^{(k)}+2x_3^{k}) \\
      x_2^{(k+1)} = \frac{1}{10}(83 + x_1^{(k+1)}+2x_3^{(k)}) \\
      x_3^{(k+1)} = \frac{1}{5}(42+x_1^{(k+1)}+x_2^{(k+1)})
    \end{array}$$
    取$x^{(0)}=(0,0,0)^T $则:$||x^{(6)}-x^*||_\infty = 0.0001 $,显然GS法收敛好于J法。

\end{example}

\subsection{J法GS法的收敛性}
收敛的充要条件:$\rho(B)<1$,充分条件:$||B||<1$

\begin{definition}
  $$|a_{ii}| >\sum^n_{\substack{j=1\\j\neq i}}|a_{ij}|, i=1,2,\dots,n $$称A为严格对角占优阵。

  $$|a_{ii}| \ge \sum^n_{\substack{j=1\\j\neq i}}|a_{ij}|, i=1,2,\dots,n $$且其中至少有一个不等式严格成立,则称A为弱对角占优阵。
\end{definition}

\begin{definition}
  设$A \in R^{n\times n}$,若不能找到排列阵P使得$P^TAP=\left[
    \begin{matrix}
      A_{11} & A_{12} \\
      0 & A_{22}
    \end{matrix}
    \right]$(其中$A_{11},A_{22} $均为方阵)成立,则称A为不可约的。
\end{definition}
\begin{example}
  $$B=\left[\begin{matrix}
    1 & 1 & 0 \\
    1 & 1 & 0 \\
    0 & 1 & 2 
  \end{matrix}\right]\rightarrow\text{1-3行交换}\left[\begin{matrix}
    0 & 1 & 2 \\
    1 & 1 & 0 \\
    1 & 1 & 0 
  \end{matrix}\right]\rightarrow\text{1,3列交换}\left[
    \begin{matrix}
      2 & 1 & 0 \\
      0 & 1 & 1 \\
      0 & 1 & 1
    \end{matrix}
    \right]
  $$
\end{example}

\begin{theorem}
  若$A=(a_{ij})_n \in R^{n\times n}$为严格对角占优阵或不可约弱对角占优阵,则$a_{ii}\neq 0, i=1,2,\dots, n$,且A为非奇异阵。
  \label{theorem:5.7}
\end{theorem}

\begin{theorem}
  若A为严格对角占优阵或不可约弱对角占优阵,则$\forall x^{(0)} $,方程$Ax=b$的J法和GS法均收敛。
\end{theorem}
\begin{proof}
  [反证法](目标$\rho(G)<1$)
  
  设G有一个特征值$\lambda$满足$|\lambda|\ge 1$,则$|\lambda I-(D-L)^{-1}U|=0\Rightarrow |I-\lambda^{-1}(D-L)^{-1}U |=0\Rightarrow |(D-L)^{-1}||D-L-\lambda^{-1}U |=0 $
  
  $\because a_{ii}\neq 0 \therefore |(D-L)^{-1} |\neq 0 $,而$A=D-L-U$与$D-L-\lambda^{-1}U$的零元素与非零元素位置完全一样,所以$D-L-\lambda^{-1}U$也是不可约的。
  
  又$\because |\lambda|\ge 1, D-L-\lambda^{-1}U $也是弱对角占优矩阵。根据定理\ref{theorem:5.7}有$|D-L-\lambda^{-1}U|\neq 0 $,矛盾,证明$\rho(B)<1$。
\end{proof}

\begin{theorem}
  设A对称,且$a_{ii}>0,i=1,2,\dots,n $则J法收敛$\Leftrightarrow$A及$2D-A$均正定。
\end{theorem}
\begin{proof}
  $\because a_{ii}>0, i=1,2,\dots, n $\\
  $\therefore D=D^{\frac{1}{2}}D^{\frac{1}{2}} $\\
  而$B=I-D^{-1}A=D^{-\frac{1}{2}}(I-D^{-\frac{1}{2}}AD^{-\frac{1}{2}} )D^\frac{1}{2} $,说明B与$I-D^{-\frac{1}{2}}AD^{\frac{1}{2}}$相似。

  \begin{itemize}
    \item 必要性:\\
      若J法收敛,则$\rho(B)<1$。设$D^{-\frac{1}{2}}AD^{-\frac{1}{2}} $的特征值$\mu$(实数)则:
      $\lambda(B) = 1-\mu, \therefore |1-\mu|<1\Rightarrow 0<\mu < 2$\\
      $\because D^{-\frac{1}{2}}AD^{-\frac{1}{2}}\text{正定} \therefore \forall x \in R^n, (D^{-\frac{1}{2}}x)^TAD^{-\frac{1}{2}}x = x^TD^{-\frac{1}{2}}AD^{-\frac{1}{2}}x >0$A正定。\\
      又$\because \lambda (2I-D^{-\frac{1}{2}}AD^{-\frac{1}{2}})=2-\mu \in (0,2)$\\
      $\therefore 2D-A$正定。
    \item 充分性:\\
      $\because A\text{正定}$\\
      $\therefore D^{-\frac{1}{2}}AD^{-\frac{1}{2}} \text{正定}$\\
      $\therefore \lambda(D^{-\frac{1}{2}}AD^{-\frac{1}{2}} )>0 $\\
      $\therefore \lambda(B)<1\ (B=I-D^{-\frac{1}{2}}AD^{-\frac{1}{2}})$\\
      又$\because -B=-D^{-\frac{1}{2}}(I-D^{-\frac{1}{2}}AD^{-\frac{1}{2}})D^{-\frac{1}{2}} 
          = D^{-\frac{1}{2}}(I-D^{-\frac{1}{2}}(2D-A)D^{-\frac{1}{2}})D^{-\frac{1}{2}}$
      $\therefore -\lambda(B)<1\Rightarrow \lambda(B) > -1$

  \end{itemize}

\end{proof}

\begin{theorem}
  设A对称正定,则方程$Ax=b$的GS法收敛。
\end{theorem}

\begin{example}
  \begin{itemize}
    \item $$\left\{
        \begin{array}{lr}
          x_1+2x_2-2x_3 = 1\\
          x_1+x_2+x_3=1 \\
          2x_1+2x_2+x_3 = 1
        \end{array}
      \right.$$J法收敛,GS法发散
    \item $$\left\{
      \begin{array}{lr}
        x_1+0.8x_2+0.8x_3 = 2.6\\
        0.8x_1+x_2+0.8x_3=2.6 \\
        0.8x_1+0.8x_2+x_3 = 2.6
      \end{array}
    \right.$$GS法收敛,J法发散
  \end{itemize}
\end{example}

\begin{example}
  当$a_{ii}>0 $,且$a_{ij}\le 0, i\neq j$时,可证下列四种情况只有一种成立。
  \begin{enumerate}
    \item $\rho(G)=\rho(B)=0$
    \item $0<\rho(G)<\rho(B)<1$
    \item $\rho(G)=\rho(B)=1$
    \item $1<\rho(B)<\rho(G)$
  \end{enumerate}
  通常情况下,GS方法好于J法,但不是所有情况。
\end{example}

\section{超松弛迭代法}
\subsection{SOR法构造}
记$\Delta x = (\Delta x_1, \Delta x_2, \dots, \Delta x_n)^T = x^{(k+1)}-x{(k)} $\\
则GS法可写成:$x^{(k+1)}=x^{(k)}+\Delta x $
其中$\Delta x_i = \frac{1}{a_{ii}}[b_i-\sum_{j=1}^{i-1}a_{ij}x_j^{(k+1)}-\sum^n_{j=i+1}a_{ij}x^{(k)}_j] - x_i^{(k)} $\\
引入参数w,即可得SOR法:
$$x^{(k+1)}=x^{(k)}+w\Delta x $$
i.e.
\begin{equation}
  x^{(k+1)}_i = (1-w)x^{(k)}_i + \frac{w}{a_{ii}}[b_i - \sum^{i-1}_{j=1}a_{ij}x_j^{(k+1)}-\sum^n_{j=i+1}a_{ij}x^{(k)}_j]
\end{equation}
将$a_{ii}$乘到等号左边,写成向量形式:
\begin{equation*}
  \begin{array}{lr}
    Dx^{(k+1)} = (1-w)Dx^{(k)}+w[b+Lx^{(k+1)}+Ux^{(k)}]\\
    \Rightarrow x^{(k+1)} = (D-wL)^{-1}[(1-w)D+wU]x^{(k)}+(D-wL)^{-1}wb
  \end{array}
\end{equation*}
令$L_w =(D-wL)^{-1}[(1-w)D+wU], f=(D-wL)wb $,则有
\begin{equation}
  x^{(k+1)}=L_wx^{(k)}+f
\end{equation}
\begin{example}
  $$\left[\begin{matrix}
    4 & 3 & 0 \\
    3 & 4 & -1 \\
    0 & -1 & 4
  \end{matrix}\right]\left[\begin{matrix}
    x_1 \\ x_2 \\ x_3 
  \end{matrix}\right]=\left[\begin{matrix}
    24 \\ 30 \\ -24
  \end{matrix}\right]$$

  精确解:$x^*=(3,4,-5)^T$ \\
  用$w=1(GS), x^{(7)}=(3.013411, 3.988824, -5.002794)^T $ \\
  $w=1.2(SOR), x^{(7)}=(3.00049, 4.000258, -5.000348)^T$
\end{example}

\begin{theorem}
  (SOR法收敛的必要条件)

  $\because$SOR法收敛,$\therefore \rho(L_w)<1$
  
  设$\lambda_1, \lambda_2, \dots, \lambda_n$为$L_w$的特征值,则$|L_w|=|\prod^n_{i=1}\lambda_i|\le |\lambda_1||\lambda_2|\dots|\lambda_n|\le [\rho(L_w)]^n< 1\Rightarrow |L_w|^{\frac{1}{n}}\le \rho(L_w)<1 $
  而$det(L_w)=det(D-wL)^{-1}det[(1-w)D+wU]=detD^{-1}det[(1-w)D]=(1-w)^n,\therefore |1-w|<1 \Rightarrow 0<w<2$
\end{theorem}

\begin{theorem}
  若A为对称正定阵,则SOR法收敛的充要条件为$0<w<2$
\end{theorem}

\begin{definition}
  若存在排列阵P使$PAP^T=\left[\begin{matrix}
    D_1 & M_1 \\
    M_2 & D_2
  \end{matrix}\right]$,其中$D_1, D_2$为对角阵,称A是2-循环的。若矩阵$\alpha D^{-1}L+\alpha^{-1}D^{-1}U $的特征值都与$\alpha$无关,则A是相容次序矩阵。
\end{definition}

\begin{theorem}
  设$A\in R^{n\times n} $是2-循环和相容次序的,$B=I-D^{-1}A $的特征值全为实数,且$\mu=\rho(B)<1$,则:
  \begin{equation*}
    \rho(L_w)=
    \left\{
      \begin{array}{lr}
        \frac{[w\mu+\sqrt{w^2\mu^2-4(w-1)}]^2}{4}, 0<w<w_{opt} \\
        w-1, w_{opt}\le w < 2
      \end{array}
      \right.
  \end{equation*}
  其中,$\rho(w_{opt})=\min\rho(L_w)=w_{opt}-1, w_{opt}=\frac{2}{1+\sqrt{1-\mu^2}} $,称为最佳松弛因子,且当$0<2<w_{opt}$时,$\rho(L_w)$是w的单减函数,当$w_{opt}\le w\le 2$时,$\rho(L_w)$是w的单增函数。

\end{theorem}

\begin{enumerate}
  \item 当w=1时,$\rho(L_1)=\mu^2=\rho(B)^2$,$R(L_w)=-ln\mu^2=2R(B)$,GS法收敛速度为J法的2倍。
  \item 显然$\rho(L_w)\ge \rho(L_{w_{opt}})=w_{opt}-1, w_{opt}\ge 1 $
  \item \begin{theorem}
    设A是对称正定的三对角阵,则:$\rho(G)=\rho(B)^2<1$,且$w_{opt}=\frac{2}{1+\sqrt{1-\rho(B)^2}} $
  \end{theorem}
\end{enumerate}

\begin{example}
  $$Ax=b, A=\left[\begin{matrix}
    4 & 3 & 0 \\
    3 & 4 & -1 \\
    0 & -1 & 4
  \end{matrix}
    \right]$$
    显然A为对称正定三对角阵(利用顺序主子式均大于0可以判断正定) \\
    $\because B=I-D^{-1}A =\left[\begin{matrix}
      & -\frac{3}{4} \\
      -\frac{3}{4} & & \frac{1}{4} \\
      & \frac{1}{4} 
    \end{matrix}\right], \therefore \rho(B)=\sqrt{5/8}\approx 0.790 < 1,\rho(G)=\rho(B)^2=0.625$ \\
    $$w_{opt}=\frac{2}{1+\sqrt{1-0.625}}\approx 1.24$$
\end{example}

\subsection{块松弛迭代法}
设$A=\left[\begin{matrix}
  A_{11} & A_{12} & \dots & A_{1N} \\
  A_{21} & A_{22} & \dots & A_{2N} \\
  \dots \\
  A_{N1} & A_{N2} & \dots & A_{NN}
\end{matrix}\right]$其中$A_{ii}$为$n_i\times n_i $的非奇异方阵,且$n_1+n_2+\dots+n_N=n$,有:
\begin{equation}
  A_{ii}x_i^{(k+1)} = (1-w)A_{ii}x_i^{(k)}+w[b_i-\sum^{i-1}_{j=1}A_{ij}x^{(k+1)}_j-\sum^n_{j=i+1}A_{ij}x^{(k)}_j]
  \label{eq:8.1}
\end{equation}
其中$x_i^{(k)}, b_i$均为$n_i$个分量的向量。

由(\ref{eq:8.1})可得:
$$x^{(k+1)}=(D-wL)^{-1}[(1-w)D+wU]x^{(k)}+w(D-wL)^{-1}b $$
其中$X=(x_1^T, x_2^T, \dots, x^T_N)^T, b=(b_1^T, b_2^T, \dots, b_N^T)^T$

\section{共轭梯度法}
\textbf{系数矩阵对称正定}

若$Ax=b$其中$A>0$,则求解可转化为求下列二次函数:
\begin{equation}
  \phi(x)=\frac{1}{2}x^TAx-x^Tb=\frac{1}{2}(Ax,x)-(b,x) 
\end{equation}
的最小值点问题。

\begin{theorem}
  设A对称正定,则$x^*$为方程组的解的充要条件是$\phi(x^*)=\min_{x\in R^n}\phi(x) $
\end{theorem}
\begin{proof}
  定义如下函数:
  $$F(x)=\frac{1}{2}(A^{-1}r, r)\ge 0 $$
  其中$r=b-Ax$,将r代入上式有:
  $$F(x)=\phi(x)+\frac{1}{2}(A^{-1}b,b) $$
  其中$\phi(x)=\frac{1}{2}x^TAx-x^Tb=\frac{1}{2}(Ax,x)-(b,x) $
  $F(x)$最小值点$\Leftrightarrow$$\phi(x)$最小值点$\Rightarrow r=0, i.e. Ax^*=b$
\end{proof}

\subsection{最速下降法}
选取初值$x^{(0)} $,则有:$-\nabla \phi(x)=-(\frac{\partial \phi}{\partial x_1}, \frac{\partial \phi}{\partial x_2}, \dots, \frac{\partial \phi}{\partial x_n})^T|_{x=x^{(0)}} =b-Ax^{(0)}=r^{(0)} $。
可令$\frac{d\phi(x^{(0)}+\alpha r^{(0)})}{d\alpha}=0$,则可得$\alpha = \frac{(r^{(0)}, r^{(0)})}{(Ar^{(0)}, r^{(0)})}=\alpha_0$,则$x^{(1)}=x^{(0)}+\alpha_0r^{(0)} $是
使得$\phi(x)$下降最快,并且使之达到最小值点的极值点,然后再从$x^{(1)}$出发,寻找一个使得$\phi(x)$下降最快的方向$r^{(1)} $和步长$\alpha_1$,同理可知:
$r^{(1)}=b-Ax^{(1)},\alpha_1=\frac{(r^{(1)}, r^{(1)})}{(Ar^{(1)}, r^{(1)})} $,则可得:$x^{(2)}=x^{(1)}+\alpha_1r^{(1)}$,依次类推可得:$x^{(3)}, x^{(4)}, \dots, x^{(k)}$,综上可得如下最速下降算法:
\begin{equation}
  \begin{array}{lr}
    r^{(k)}=b-Ax^{(k)} \\
    \alpha_k = \frac{(r^{(k)}, r^{(k)})}{(Ar^{(k)}, r^{(k)})} \\
    x^{(k+1)}=x^{(k)}+\alpha_kr^{(k)}
  \end{array}
\end{equation}
易证:$\phi(x^{(k+1)}) < \phi(x^{(k)}),|x^{(k+1)}-x^{(k)}|<\varepsilon$

\subsection{共轭梯度法}
对于$A>0$称满足$(AP^{(i)}, P^{(j)})=0, i\neq j$的向量组$\{P^{(i)}\}$为A共轭向量组,如果按方向$P^{(0)}, P^{(1)}, \dots, P^{(k-1)}$已进行了k次一维搜索,求得了$x^{(k)} $,下一步求
$x^{(k+1)} $,则需要进行一维搜索使$\phi(x^{(k)}+\alpha P^{(k)}) $极小,则可令$\frac{d\phi(x^{(k)}+\alpha p^{(k)})}{d\alpha}=0 \Rightarrow \alpha_k = \frac{(r^{(k)}, p^{(k)})}{(Ap^{(k)}, p^{(k)})}$

由此可得:$x^{(k+1)}=x^{(k)}+\alpha_kp^{(k)}, r^{(k+1)}=b-Ax^{(k+1)}=r^{(k)}-\alpha_kAP^{(k)} $,其中$(AP^{(i)}, P^{(j)})=0, i\neq j, p^{(0)}=r^{(0)}=b-Ax^{(0)} $

按此方法有如下性质:
\begin{equation*}
  \begin{array}{lr}
    \phi(x^{(k+1)})=\min_\alpha\phi(x^{(k)}+\alpha p^{(k)}) \\
    \phi(x^{(k+1)}) = \min_{x\in \spn\{p^{(0)}, p^{(1)}, \dots, p^{(k)} \}} \phi(x)
  \end{array}
\end{equation*}

开始时选取$p^{(0)}=r^{(0)}$,然后选取$p^{(k)}=r^{(k)}+\beta_{k-1}p^{(k-1)} $,其中$\beta_{k-1} $由A-共轭性确定,i.e.,由
$(Ap^{(k)}, p^{(k-1)})=0$确定为:$\beta_{k-1}=-\frac{(r^{(k)}, Ap^{(k-1)})}{(p^{(k-1)}, Ap^{(k-1)})} $

综上可得如下CG算法:
给定:$x^{(0)},p^{(0)}=r^{(0)}, r^{(0)}=b-Ax^{(0)} $
\begin{align}
  &\alpha_k = \frac{(r^{(k)}, r^{(k)})}{(Ap^{(k)}, p^{(k)})} \\
  &x^{(k+1)}=x^{(k)}+\alpha_k p^{(k)}  \\
  &r^{(k+1)}=r^{(k)}-\alpha_kAp^{(k)}\\
  &\beta_k = \frac{(r^{(k+1)}, r^{(k+1)})}{(r^{(k)}, r^{(k)})} \\
  &p^{(k+1)}=r^{(k+1)}+\beta_k p^{(k)}
\end{align}

\end{document}
